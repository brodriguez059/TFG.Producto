% ################################################################
% #######     INSERTABLES                           ##############
% ################################################################

% ************************* Beatiful colorboxes (breakable) ********************************

% ******************************* Tables and lists *********************************

\newcolumntype{P}[1]{
    >{\RaggedRight\hangafter=1\hangindent=1em}m{#1}
} % automatic hanging indentations

\newcolumntype{J}{
    >{\justifying\arraybackslash}X
}
\newcolumntype{M}[1]{
    >{\Centering\arraybackslash}m{#1}%{\dimexpr.2\linewidth-2\tabcolsep}
} %Columa especial
% \newcolumntype{N}{>{\justifying\arraybackslash}m{\dimexpr.2\linewidth-2\tabcolsep}} %Columa especial
% \newcolumntype{L}{>{\RaggedRight\arraybackslash}X}

% \newcolumntype{L}[1]
% {%
%     >{\RaggedRight\let\newline\\\arraybackslash\hspace{0pt}}m{#1}
% }
% \newcolumntype{C}[1]
% {%
%     >{\Centering\let\newline\\\arraybackslash\hspace{0pt}}m{#1}
% }
% \newcolumntype{R}[1]
% {%
%     >{\RaggedLeft\let\newline\\\arraybackslash\hspace{0pt}}m{#1}
% }

\setlength{\tabcolsep}{3ex}
\renewcommand{\arraystretch}{1.25}

% *************************** Listings *****************************

% \DeclareCaptionFont{white}{\bfseries\color{white}}
% \DeclareCaptionFormat{listing}{\colorbox{gray}{\parbox{\dimexpr\linewidth-2\fboxsep\relax}{#1#2#3}}} % Color box de color gris y ajustado correctamente al tamaño del listing
% \captionsetup[lstlisting]{format=listing,labelfont=white,textfont=white}

\lstset{
    aboveskip={1.5\baselineskip},
    backgroundcolor=\color{backcolour}, % choose the background color. You must add \usepackage{color}
    basicstyle=\scriptsize\ttfamily,         % the size of the fonts that are used for the code
    breakatwhitespace=false,            % sets if automatic breaks should only happen at whitespace
    breaklines=true,                    % sets automatic line breaking
    captionpos=b,                       % sets the caption-position to (b)ottom/(t)op
    commentstyle=\color{commentcolour}, % comment style
    columns=fullflexible,                      %
    deletekeywords={...},               % if you want to delete keywords from the given language
    emph={SCORE,CODE,ID,LEMA,POS},
    emphstyle=\color{light-brown},
    emphstyle={[2]\color{blue}},
    escapeinside={\%*}{*)},             % if you want to add a comment within your code
    extendedchars = true,               % Extended ASCII
    firstnumber=1,                      % start line enumeration with line 1
    float=[*],
    frame=single,                       % adds a frame around the code
	framesep=3pt,
    framerule=0.6pt,
    framexleftmargin=1pt,
    identifierstyle=\ttfamily,
    inputencoding = utf8,               % Input encoding
    keepspaces=true,                    % keeps spaces in text, useful for keeping indentation of code (possibly needs columns=flexible)
    keywordstyle=\color{keywordcolour}, % keyword style
    lineskip=0pt,
	linewidth=0.98\linewidth,
	moredelim=[il][\sffamily\scriptsize\slshape\itshape\color{GRISARGIA}]{º},
	moredelim=[is][\bfseries]{ª}{ª},
    moreemph={[2]top,num,ENtitle,TERM,WF,SYNSET,ENdesc,ENnarr,EStitle,ESdesc,ESnarr,EXP,DOC,DOCNO,DOCID,HEADLINE,TEXT},
    morekeywords={*,SCORE,...},               % if you want to add more keywords to the set
    numbers=left,                       % where to put the line-numbers
    numberstyle=\tiny\color{gray},      % the style that is used for the line-numbers
    numbersep=5pt,                      % how far the line-numbers are from the code
    postbreak=\mbox{\textcolor{red}{$\hookrightarrow$}\space}, % Adds arrow after break
    rulecolor=\color{black},            % if not set, the frame-color may be changed on line-breaks within not-black text (e.g. comments (green here))
    showspaces=false,                   % show spaces adding particular underscores
    showstringspaces=false,             % underline spaces within strings
    showtabs=false,                     % show tabs within strings adding particular underscores
    stepnumber=1,                       % the step between two line-numbers. If it's 1, each line will be numbered
    stringstyle=\color{stringcolour}\ttfamily,          % string literal style
    tabsize=4,                          % sets default tabsize to 2 spaces
    title=\lstname,                     % show the filename of files included with \lstinputlisting; also try caption instead of title
    upquote=true,
    xleftmargin=5pt,
}

\lstset{literate  =        % Support additional characters (Allows the use of many UTF8 characters)
      {á}{{\'a}}1 {é}{{\'e}}1 {í}{{\'i}}1 {ó}{{\'o}}1 {ú}{{\'u}}1
      {Á}{{\'A}}1 {É}{{\'E}}1 {Í}{{\'I}}1 {Ó}{{\'O}}1 {Ú}{{\'U}}1
      {à}{{\`a}}1 {è}{{\`e}}1 {ì}{{\`i}}1 {ò}{{\`o}}1 {ù}{{\`u}}1
      {À}{{\`A}}1 {È}{{\'E}}1 {Ì}{{\`I}}1 {Ò}{{\`O}}1 {Ù}{{\`U}}1
      {ä}{{\"a}}1 {ë}{{\"e}}1 {ï}{{\"i}}1 {ö}{{\"o}}1 {ü}{{\"u}}1
      {Ä}{{\"A}}1 {Ë}{{\"E}}1 {Ï}{{\"I}}1 {Ö}{{\"O}}1 {Ü}{{\"U}}1
      {â}{{\^a}}1 {ê}{{\^e}}1 {î}{{\^i}}1 {ô}{{\^o}}1 {û}{{\^u}}1
      {Â}{{\^A}}1 {Ê}{{\^E}}1 {Î}{{\^I}}1 {Ô}{{\^O}}1 {Û}{{\^U}}1
      {ã}{{\~a}}1 {ẽ}{{\~e}}1 {ĩ}{{\~i}}1 {õ}{{\~o}}1 {ũ}{{\~u}}1
      {Ã}{{\~A}}1 {Ẽ}{{\~E}}1 {Ĩ}{{\~I}}1 {Õ}{{\~O}}1 {Ũ}{{\~U}}1
      {œ}{{\oe}}1 {Œ}{{\OE}}1 {æ}{{\ae}}1 {Æ}{{\AE}}1 {ß}{{\ss}}1
      {ű}{{\H{u}}}1 {Ű}{{\H{U}}}1 {ő}{{\H{o}}}1 {Ő}{{\H{O}}}1
      {ç}{{\c c}}1 {Ç}{{\c C}}1 {ø}{{\o}}1 {å}{{\r a}}1 {Å}{{\r A}}1
      {€}{{\euro}}1 {£}{{\pounds}}1 {«}{{\guillemotleft}}1
      {»}{{\guillemotright}}1 {ñ}{{\~n}}1 {Ñ}{{\~N}}1 {¿}{{?`}}1 {¡}{{!`}}1 
      % ¿ and ¡ are not correctly displayed if inconsolata font is used
      % together with the lstlisting environment. Consider typing code in
      % external files and using \lstinputlisting to display them instead. 
}%

\lstdefinestyle{consola}
{
    numbers=none,
    xleftmargin=\parindent,
    xrightmargin=\parindent,
    aboveskip=3mm,
    belowskip=0.01mm,
    basicstyle=\scriptsize\bf\ttfamily,
    backgroundcolor=\color{gray75}
}

\lstdefinestyle{no_fileconf}
{
    numbers=none,
    xleftmargin=\parindent,
    xrightmargin=\parindent,
    aboveskip=3mm,
    belowskip=0.01mm,
    basicstyle=\footnotesize\ttfamily,
    backgroundcolor=\color{gray90},
}

\lstdefinestyle{fileconf}
{
    xleftmargin=\parindent,
    xrightmargin=\parindent,
    aboveskip=3mm,
    belowskip=0.01mm,
    basicstyle=\footnotesize\ttfamily,
    backgroundcolor=\color{gray95},
}

\lstnewenvironment{listing}[1][]{\lstset{#1}\pagebreak[0]}{\pagebreak[0]}

% ************************* Margin Notes ********************************

% *********************************** TODO Comments *********************************
% \setlength {\marginparwidth }{2cm}
\newcommandx{\urgent}[2][1=]{\todo[linecolor=red,backgroundcolor=red!25,bordercolor=red,#1]{#2}}
\newcommandx{\change}[2][1=]{\todo[linecolor=yellow,backgroundcolor=yellow!25,bordercolor=yellow,#1]{#2}}
\newcommandx{\unsure}[2][1=]{\todo[linecolor=blue,backgroundcolor=blue!25,bordercolor=blue,#1]{#2}}
\newcommandx{\improvement}[2][1=]{\todo[linecolor=Plum,backgroundcolor=Plum!25,bordercolor=Plum,#1]{#2}}
\newcommandx{\info}[2][1=]{\todo[linecolor=OliveGreen,backgroundcolor=OliveGreen!25,bordercolor=OliveGreen,#1]{#2}}