\chapter{Marco Teórico}

\section{Simulación}\label{sec:simulacion}

%% Todavía no se ha explicado lo que es la simulación de sistemas

% Bloque simulación de sistemas

% \change[inline]{Definición de simulación}
% \change[inline]{Explicación sobre simulación de sistemas.}
% \change[inline]{¿Por qué y para qué la simulación de sistemas?}
% \change[inline]{Puedes citar a uno de tus libros aquí.}

\subsection{Definiciones}

\subsubsection{Modelo}

% \change[inline]{Haz una cita a la definición de tu profesor de simulación de
% sistemas (igual no se puede)}
% %https://normas-apa.org/referencias/citar-curso-o-material-de-clase/

% \change[inline]{Definición de modelo (Nos centraremos en modelos
% probabilísticos)}

% \change[inline]{¿Clasificación de modelos? (los modelos de simulación deben
% ser simbólicos (no tienen una relación física o analógica con el sistema real,
% sino una relación lógica))}

% \change[inline]{¿Qué significa ser dinámico?}
% \change[inline]{¿Qué significa ser discreto?}

\subsubsection{Sistema}
% \change[inline]{Definición de sistema}
% \change[inline]{Partes y conceptos de un sistema de simulación (entorno del
% sistema, entidad, atributo, actividad, estado, suceso, término endógeno,
% término exógeno, contadores estadísticos, medidas de rendimiento)}

\subsection{Ventajas e inconvenientes}
% \change[inline]{Ventajas e inconvenientes}

\subsection{Diseño basado en eventos}
% \change[inline]{Grafo de sucesos}

\subsection{Sistemas dinámicos discretos}
% \change[inline]{Partes fundamentales (en común) de los simuladores dinámicos
% discretos (lista de sucesos, temporizador de eventos, reloj...)}

\section{Compilación}\label{sec:compilacion}

% Bloque de compilación

% \change[inline]{Habla aquí de los procesadores del lenguaje (formal)}
% \change[inline]{¿Qué significa compilar?}
% \change[inline]{Definición de Transpilador}

\subsection{Conceptos básicos}

% \subsection{Analizador léxico}
% \change[inline]{Lenguajes regulares}
% \change[inline]{Expresiones regulares}
% \change[inline]{Tokens}

\subsubsection{Analizador sintáctico}
% \change[inline]{Lenguajes libres de contexto}
% \change[inline]{Gramática libre de contexto}
% \change[inline]{ETDS y acciones}

\subsubsection{Analizador semántico}
% \change[inline]{Tabla de símbolos}
% \change[inline]{Verificación estática}

\subsection{Herramientas}
% \change[inline]{Compilador C++}
% \change[inline]{Lex y Flex}
% \change[inline]{Yacc y Bison}