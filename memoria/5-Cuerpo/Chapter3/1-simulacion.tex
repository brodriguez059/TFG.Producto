% Bloque simulación de sistemas

\section{Simulación}\label{sec:simulacion}

% \change[inline]{Introducción a la simulación de sistemas.}
A lo largo del documento hemos estado mencionando la simulación de sistemas,
pero no hemos definido exactamente qué es.

\change[inline]{La simulación es una potente técnica de resolución de problemas}
\change[inline]{Orígenes: La teoría de muestreo estadístico y el análisis
probabilístico de complejos sistemas físicos}
\change[inline]{Punto en común: uso de número aleatorios y muestreo aleatorio
para aproximar un resultado de una solución}
\change[inline]{Primera aplicación: Por Von Neuman y Ulam, durante la segunda
guerra mundial, estudio de problemas de difusión aleatoria d\change[inline]{}e neutrons, en
conexión con el desarrollo de la bomba atómica}
\change[inline]{El proyecto era alto secreto, se le dio un nombre clave: Monte
Carlo, en referencia al famoso casino de juego}
\change[inline]{El nombre persistió durante un tiempo como sinónimo de cualquier
simulación pero hoy en día está restringido a una rama de la matemática
experimental que trata con números aleatorios (y que puede relacionarse con lo
que llamaremos modelos de simulación estáticos)}
\change[inline]{Mientras que el término simulación, o simulación de sistemas, se
refiere a una técnica de análisis más extensa, aunque muy a menudo utiliza
números aleatorios}

\change[inline]{La contribución individual más importante es la potencia de cálculo y velocidad de procesamiento de los ordenadores}


% \urgent[inline]{Definición de simulación}

\change[inline]{Definición informal:}

\begin{quote}
    Una simulación es la imitación del modo de funcionamiento u operación de un
    proceso o sistema del mundo real. La simulación involucra la generación de
    una historia artifical de un sistema, y la observación de esa historia
    artificial para obtener inferencias relativas a las características de
    funcionamiento de dicho sistema.
\end{quote}\change{Cambios en esta redacción}


\change[inline]{Definición formal:}

\change[inline]{Nosotros hablaremos de una definición más exacta de “Simulación de Sistemas” y
“Modelado” más adelante, pero por ahora podemos citar la definición encontrada
en } % Usemos los apuntes aquí

\begin{quote}
    El proceso de diseñar un modelo lógico o matemático de un sistema real y
    realizar experimentos basados en el ordenador con el modelo, al objeto de
    descrbir, explicar o predecir el comportamiento del sistema real.
\end{quote}\change{Cambios en esta redacción}

\change[inline]{
Actualmente hay muchos tipos distintos de sistemas de simulación, como por
ejemplo los “Modelos de Montecarlo” y “Modelos en ecuaciones diferenciales”. No
obstante, el principal objeto de estudio de este proyecto será una categoría muy
importante a la que se conoce como “Modelos de simulación dinámicos discretos”,
especificamente aquellos que se pueden modelar a través de un diseño orientado a
eventos.
}

Se hace uso de otras técnicas:
\change[inline]{La modelización permite obtener una representación abstracta del sistema real}
\change[inline]{Las técnicas de programación de ordenadores: El programa de ordenador traduce el modelo a una forma operativa}
\change[inline]{La teoría de la probabilidad define las variables aleatorias del modelo, y ayuda a construir las historias artificaciones (generadores de datos) necesarios}
\change[inline]{La estadística ayuda en el diseño y análisis de experimentos a realizar para obtener respuestas}
\change[inline]{Los métodos heurísticos se emplean para conseguir buenas soluciones, sino óptimas}

% \change[inline]{¿Por qué y para qué la simulación de sistemas?}
\change[inline]{
El modelado matemático o modelado analítico se aprovecha de las características
del problema cuya respuesta se desea obtener para llegar a la mejor conclusión.
Sin embargo, muchas veces nos encontraremos con problemas que no se pueden
modelar analíticamente debido a su complejidad en tiempo o espacio. En estas
situaciones, debemos hacer uso de técnicas de aproximación de resultados para
dar con soluciones que, aunque no se pueden garantizar óptimas, sí que se pueden
considerar lo suficientemente buenas. Una de estas técnicas vendría a ser la
simulación de sistemas, la cual a través de modelado simbólico/lógico intenta
reproducir el comportamiento de un determinado sistema con el fin de analizar
una serie de resultados a escoger.
}

\change[inline]{La simulación tiene sentido en la medida en que un ordenador permite plantear, describir y condificar modelo de grandes y complejos sistemas, que no se pueden resolver por el cálculo matemático estadístico de forma útil}
\change[inline]{Se abre la posibilidad de, al igual que por ejemplo, los físicos y biólogos, realizar experimentos "de laboratorio" en áreas donde tradicionalmente esto no se podía hacer}
\change[inline]{Esto representa una gran ventaja en cuanto a las posibilidades de controlar las condiciones de tales experimentos y en consecuencia incrementar la información obtenida de los mismos}


\subsection{Definiciones}

\subsubsection{Modelo}

% \change[inline]{Haz una cita a la definición de tu profesor de simulación de
% sistemas (igual no se puede)}
% %https://normas-apa.org/referencias/citar-curso-o-material-de-clase/

\change[inline]{Un modelo es una representación o una abstracción de un sistema
con el propósito de estudiar tal sistema, pero que contiene sólo lo esencial del
sistema real}
\change[inline]{Aquellos aspectos del sistema que no contribuyen de forma
significativa al comportamiento del sistema no están incluidos en el modelo. Por
tanto un modelo es: un substituto del sistema real, una simplificación del mismo}
\change[inline]{Un modelo probabilístico }

% \change[inline]{Definición de modelo (Nos centraremos en modelos
% probabilísticos)}

% \change[inline]{¿Clasificación de modelos? (los modelos de simulación deben
% ser simbólicos (no tienen una relación física o analógica con el sistema real,
% sino una relación lógica))}

% \change[inline]{¿Qué significa ser dinámico?}
% \change[inline]{¿Qué significa ser discreto?}

\subsubsection{Sistema}
% \change[inline]{Definición de sistema}
% \change[inline]{Partes y conceptos de un sistema de simulación (entorno del
% sistema, entidad, atributo, actividad, estado, suceso, término endógeno,
% término exógeno, contadores estadísticos, medidas de rendimiento)}

\subsection{Ventajas e inconvenientes}
% \change[inline]{Ventajas e inconvenientes}


\subsection{Diseño basado en eventos}
% \change[inline]{Grafo de sucesos}

\subsection{Sistemas dinámicos discretos}
% \change[inline]{Partes fundamentales (en común) de los simuladores dinámicos
% discretos (lista de sucesos, temporizador de eventos, reloj...)}

