\section{Alcance}\label{sec:alcance}

El alcance de este proyecto incluye el trabajo necesario para diseñar,
implementar y documentar un framework y una aplicación de modelado y desarrollo
de simulación de sistemas dinámicos discretos basados en eventos. La división
del trabajo como tal se hará a través de los siguientes tres módulos principales:
\begin{itemize}
    \item \textbf{Lenguaje}: Un lenguaje de modelado específico pensado para ser
    usado dentro de la propia aplicación del producto por usuarios que no tengan
    mucha experiencia en programación. Este lenguaje se usará sólo en
    situaciones específicas dentro del flujo de eventos para indicar conjuntos
    de expresiones simples que se deben evaluar. Se hará uso de las herramientas
    de desarrollo de compiladores Flex y Bison para generar un transpilador que
    traduzca ficheros de este lenguaje a código Python. A través de él se
    plantea:
    \begin{itemize}
        \item Permitir que el programador del lenguaje se encargue sólo de
        realizar las especificaciones pertinentes a algunas partes de la
        simulación que se desee desarrollar:
        \begin{itemize}
            \item Especificación de expresiones, atributos y valores a usar
            en distintas partes de la simulación.
            \item Especificación de las métricas de salida de la simulación a
            desarrollar.
            \item Especificación de los generadores de datos asociados a la
            simulación a crear.
            \item Especificación de las acciones asociadas a los eventos del
            simulador.
            \item Especificación de la condición de parada del simulador.
        \end{itemize}
    \end{itemize}
    \item \textbf{Núcleo}: Una serie de módulos que implementarán un microframework de
    simulación de este tipo de sistemas en específico para Python, pensado para
    ser usado por programadores y acotar la traducción del nuevo lenguaje. A
    través de él se plantea:
    \begin{itemize}
        \item Permitir que la traducción del lenguaje incluya dentro del fichero
        generado las estructuras de datos, funciones y procedimientos que tienen
        en común todos los sistemas dinámicos discretos: \urgent{Aquí te falta
        la posibilidad de poner las cosas variables también}
        \begin{itemize}
            \item Generadores de datos aleatorios para distintos tipos de
            distribuciones.
            \item Reloj y temporizador de simulación para ejecutar los eventos.
            \item Estructura de datos para almacenar los sucesos según deben
            ocurrir en el tiempo.
            \item Las respectivas implementaciones mínimas de los dos eventos
            que siempre formarán parte de todos los modelos: “Inicio” y “Fin”.
            \item Generador de informes final que se ejecutará al finalizar la
            simulación y mostrará los resultados que se deseaban estudiar con
            ésta.
        \end{itemize}
    \end{itemize}
    \item \textbf{CLI}: Una interfaz de comandos por terminal que se usará para
    gestionar, configurar y ejecutar los proyectos desarrollados con este
    framework. \improvement{La utilización de una CLI ahora no
    resulta viable}
\end{itemize}


% -----------------------------------------------------------------------------


\subsection{Objetivos}\label{subsec:objetivos}

\subsubsection{Objetivo general}
Generar un lenguaje de modelado de simulación de sistemas dinámicos discretos
basados en eventos junto con un transpilador que lo traduzca a Python, un
microframework para acotar la traducción y un CLI para gestionar proyectos
desarrollados con este producto. \improvement{La utilización de una CLI ahora no
resulta viable}

\subsubsection{Objetivos específicos}
\begin{enumerate}
    \item Diseñar un nuevo lenguaje de modelado y simulación de sistemas
    dinámicos discretos basados en eventos usando las herramientas de desarrollo
    de procesadores de lenguaje Flex y Bison.\improvement{Se necesitan hacer
    cambios en cuanto a las herramientas a usar. Existe una herramienta que
    aplica Flex/Bison dentro de Python como tal}
    \item Diseñar, implementar y verificar una serie de módulos y
    funcionalidades realizadas en Python con el fin de generar un microframework
    para simular estos mismos sistemas.
    \item Implementar y verificar un transpilador que traduzca el lenguaje del
    producto a Python usando las características del objetivo anterior.
    \item Diseñar, implementar y verificar una serie de operaciones accesibles
    desde una CLI de cara a ser usadas para la gestión, configuración y
    ejecución parametrizada de simulaciones realizadas con el producto. \improvement{La utilización de una CLI ahora no
    resulta viable}
    \item Implementar distintas técnicas de análisis de salidas, experimentación
    y optimización de modelos dentro del núcleo del framework.
    \item Desarrollar un manual de usuario que contendrá toda la documentación
    necesaria para hacer uso del framework.
\end{enumerate}


% -----------------------------------------------------------------------------


\subsection{Requisitos}\label{subsec:requisitos}
El requisito base principal del proyecto consiste en cumplir con un tiempo de
dedicación total máximo de 300 horas.

\subsubsection{Requisitos funcionales}
\urgent{Requisitos funciones = lo que hace la aplicación. Sobre la funcionalidades, qué hace, lo que hace, cómo lo tiene que hacer}
\begin{itemize}
    \item El producto consistirá de una interfaz gráfica a modo de aplicación de escritorio usando Python y Qt.
    \item Procesará los dos eventos mínimos del grafo de eventos de la simulación: \emph{Inicio} y \emph{Final}.
    \item Los dos eventos mínimos del grafo de eventos no se podrán eliminar, sólo configurar.
    \item Se podrán añadir y eliminar eventos del grafo de eventos
    \item Se verificará que los valores de las métricas de la simulación y de los generadores de datos cumplan con sus restricciones de tipo y rango.
    \item Se permitirá tanto la ejecución de la simulación dentro de la interfaz como la
    generación de un fichero ejecutable en Python que será independiente de la
    aplicación y hará uso del framework a genera.
\end{itemize}
\subsubsection{Requisitos no-funcionales}
\urgent{TERMINAR}
\begin{markdown}
* El sistema no puede devolver datos erróneos a excepción que se aclare que es una versión beta y que los datos no son de fiar, blabla (fiabilidad)
* El código fuente está disponible a través de un repositorio github bajo licencia blabla (disponibilidad)
* La interfaz está diseñada siguiendo los diez principios de usabilidad heurística (usabilidad)
* Él sistema es responsable de verificar los datos que le ingresan como prueba para evitar errores de ejecución (robustez)
* se verifican los flujos de eventos para evitar cualquier error de interfaz del usuario
* como no se requiere registro para el uso de la aplicación, no se almacenarán datos personales ni de ningún otro tipo que comprometan la privacidad y seguridad del usuario (seguridad)
* el software está creado de tal forma que permite su continuo desarrollo en términos de funcionalidades, utilizando clases y documentando blabla (escalabilidad) 
* el programa puede ser ejecutado en plataformas Linux versiones Debian, Ubuntu, …, Windows a partir de su versión 8…, Mac OS a partir de su versión OS Mavericks. Blabla (portabilidad)
* Los nuevos desarrollos mantendrán backward compatibility (mantenibilidad)
* La aplicación se construye en módulos independientes funcionalmente que permite utilizar funciones en otros lugares (reusabilidad)
* Se defineuna tabla con el patrón de nombres de variables, constantes, funciones, clases, etc y toda la aplicación sigue esa sintaxis (legibilidad)
* El idioma de la aplicación es inglés? español?
* Los tipos ya definidos son Var, int…
* La aplicación tiene un tiempo promedio de respuesta de no más de xx segundos y maneja hasta un juego de xxx datos (rendimiento)
\end{markdown}

% Sin embargo, se pueden listar otros requisitos específicos:

% \subsubsection{Caracterización de la memoria}
% \begin{itemize}
%     \item Debe ser bien citado y referenciado.
%     \item Apartado de calidad.
% \end{itemize}

% \subsubsection{Caracterización del framework}
% \begin{itemize}
%     \item Un lenguaje verificado que no pete a la primera
%     \item El resultado son métricas en un dataframe (o varios si harás lo de
%     validación)
%     \item Cumplir con la línea base de calidad definida en el apartado (...)
% \end{itemize}

% \subsubsection{Caracterización del manual}
% \begin{itemize}
%     \item Utilizar referencias a recursos ajenas
% \end{itemize}

% \subsubsection{Licencia del producto}


% -----------------------------------------------------------------------------


% \subsection{Entregables}

% \begin{itemize}
%     \item Memoria: \change{Relacionados con el objeto de proyecto en sí}
%     \item Framework: \change{Relacionados con el objeto de proyecto en sí}
%     \item Manual: \change{Relacionados con el objeto de proyecto en sí}
% \end{itemize}


% -----------------------------------------------------------------------------


% \subsection{Exclusiones}

% Mi proyecto devolverá los datos en un sólo formato, no devolveré más tipos de
% ficheros para darle gusto al usuario.


% -----------------------------------------------------------------------------


% \subsection{Supuestos}

% \urgent[inline]{}


% -----------------------------------------------------------------------------


\subsection{EDT}\label{subsec:edt}

\begin{figure}[H]
    \centering
    \includegraphics[width=0.85\linewidth]{5-Cuerpo/Chapter2/EDT.png}
    \caption[EDT del Proyecto]{Esquema de Descomposición de Trabajo del Proyecto}\label{fig:EDT}
\end{figure}

\subsubsection{Rama de Gestión}
El paquete de trabajo \textbf{Gestión (Ge)} contendrá:
\begin{itemize}
    \item \textbf{Planificación (Ge.P):} Este paquete de trabajo agrupará todas
    las tareas relacionadas a la realización de la planificación y preparación
    inicial del proyecto.
    \item \textbf{Seguimiento y Control (Ge.S):} Este paquete de trabajo agrupará
    todas las tareas necesarias para asegurar que el seguimiento del proyecto
    está realizando como se plantea en la planificación y, en caso de no ser
    así, controlar las consecuencias de las desviaciones emergentes.
\end{itemize}

\subsubsection{Rama de Formación e Investigación}
El paquete de trabajo \textbf{Formación e Investigación (FeI)} contendrá todas
las tareas relacionadas con el aprendizaje de herramientas, búsqueda de
referencias y recolección de información necesaria para el desarrollo del
proyecto.

\subsubsection{Rama de Producto (Pr)}
El paquete de trabajo \textbf{Producto (Pr)} contendrá todas las tareas
relacionadas al propio diseño, implementación y verificación de los entregables
principales del proyecto. Podemos desglosarlo en tres paquetes más pequeños:
\begin{itemize}
    \item \textbf{Memoria (Pr.Me):} Este paquete de trabajo agrupará todas las
    tareas necesarias para la realización de la memoria final del proyecto.
    \item \textbf{Framework (Pr.Fr):} Este paquete de trabajo agrupará todas las
    tareas relacionadas al desarrollo del framework planteado en el alcance del
    proyecto. Podemos separarlo en sus tres partes principales:
    \begin{itemize}
        \item \textbf{Lenguaje (Pr.Fr.Le):} Este paquete contendrá todas las
        tareas relacionadas al diseño y desarrollo del nuevo lenguaje de
        modelado.
        \item \textbf{Core (Pr.Fr.Co):} Este paquete contendrá todas las
        tareas relacionadas al diseño y desarrollo del propio núcleo del
        framework.
        \item \textbf{CLI (Pr.Fr.Cli):} Este paquete contendrá todas las
        tareas relacionadas al diseño y desarrollo de la interfaz de comandos
        proporcionada para facilitar la gestión de proyectos realizados con el
        a generar. \improvement{La utilización de una CLI ahora no
        resulta viable}
    \end{itemize}
    \item \textbf{Manual (Pr.Ma):} Este paquete de trabajo agrupará todas las
    tareas necesarias para la realización del manual de usuario que se generará
    para facilitar el uso del producto a desarrollar.
\end{itemize}

\subsubsection{Rama de Instalación y Paquetes}
El paquete de trabajo \textbf{Instalación y Paquetes (IyP)} agrupará todas las
tareas relacionadas a la preparación del entorno de desarrollo y la instalación
de paquetes y dependencias para la realización del producto principal del
proyecto.