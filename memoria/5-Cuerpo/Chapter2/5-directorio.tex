\section{Sistema de información y comunicaciones}


% -----------------------------------------------------------------------------


\subsection{Sistema de Información}

\subsubsection{Estructura}

\improvement{Aquí debemos explicar la estructura}

\textbf{TFG:} Directorio principal del proyecto. Se divide en:

\begin{itemize}
    \item \textbf{TFG.Proyecto:} Aquí residen los entregables del proyecto.
    \begin{itemize}
        \item \textbf{Memoria:} Los ficheros relacionados a la memoria
        \item \textbf{Framework:} Los ficheros relacionados al framework
        \begin{itemize}
            \item \textbf{Lenguaje:} Los ficheros relacionados al lenguaje del framework
            \item \textbf{Core:} Los ficheros relacionados al núcleo del framework
            \item \textbf{CLI:} Los ficheros relacionados al CLI del framework\improvement{La utilización de una CLI ahora no
            resulta viable}\improvement{EXISTE LA NECESIDAD DE CAMBIAR DE UN CLI A UNA GUI}
        \end{itemize}
        \item \textbf{Manual:} Los ficheros relacionados al manual
    \end{itemize}
    \item \textbf{TFG.Gestión:} Aquí residen los ficheros relacionados a la gestión del proyecto
    \begin{itemize}
        \item \textbf{Planificación:} Los ficheros relacionados a la planificación
        \item \textbf{Seguimiento:} Los ficheros relacionados al seguimiento y control
        \item \textbf{Actas:} Las actas de cada reunión realizada
    \end{itemize}
\end{itemize}

% \subsubsection{Formato}

% \subsubsection{Denominación}

\subsubsection{Copias de seguridad}
\change[inline]{Usamos GitHub para control de versiones y copias de seguridad}
\change[inline]{Todo está dividido en varios módulos y submódulos.}
\change[inline]{Los enlaces a todo están en: \url{https://github.com/brodriguez059/TFG}}
\change[inline]{Usamos Drive para generar las actas de fin de iteración}


% -----------------------------------------------------------------------------


\subsection{Comunicaciones}

\begin{itemize}
    \item (Reuniones a través de Webex)
    \item (Comunicaciones para responder dudas a través de email)
\end{itemize}