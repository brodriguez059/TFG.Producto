\section{Análisis de riesgos y viabilidad}

\subsection{Análisis de riesgos}

\subsubsection{Métricas}

% PMBOK - Project Management Body of Knowledge 6º edición

{
\setlength{\tabcolsep}{2ex} % Table column separation
\renewcommand{\tabularxcolumn}[1]{>{\arraybackslash}m{#1}} % Default column type
\renewcommand\arraystretch{1.2} %
\rowcolors{2}{gray!25}{white} % Alternating row colors
\begin{xltabular}{\linewidth}
    {
    % @{}
        M{0.13\linewidth}
        M{0.1\linewidth}|
        |M{0.1\linewidth}
        M{0.1\linewidth}
    % @{}
    }
    \toprule        % table caption, ref label
        \rowcolor{white}
        \textbf{Probabilidad} &
        \textbf{Valor} &
        \textbf{Impacto} &
        \textbf{Valor} \\
    \midrule        % line head body
    \endfirsthead   % Definition of 1. table header

    \rowcolor{white!0}
    \multicolumn{4}{c}{Continuación de la página anterior}\\
    \toprule
        \rowcolor{white}
        \textbf{Probabilidad} &
        \textbf{Valor} &
        \textbf{Impacto} &
        \textbf{Valor} \\
    \midrule        % line head body
    \endhead        % Delongtab1finition of all following headers

    \midrule
    \rowcolor{white!0}
    \multicolumn{4}{c}{Continua en la siguiente página}\\ % footer 1. (and more) part(s) of table
    \endfoot        % foots of the table without the last one

    \bottomrule
    \rowcolor{white!0}
    \caption{Métricas para el análisis de riesgos \label{tab:riesgos-metricas}}
    \endlastfoot    % the last(!!) foot of the table
    % Data:
    Nada probable         & 0.10           & Muy bajo         & 0.05 \\
    Poco probable         & 0.30           & Bajo             & 0.10 \\
    Probable              & 0.50           & Medio            & 0.20 \\
    Muy probable          & 0.70           & Alto             & 0.40 \\
    Casi probable         & 0.90           & Muy alto         & 0.80 \\
\end{xltabular}
}

\subsubsection{Tabla de riesgos}
% \begin{enumerate}
%     \item Mala gestión del tiempo (falta de tiempo + pérdida de tiempo)
%     \item Problemas de contacto con [directora] (por problemas de conexión,
%     enfermedad, etc. Tanto en el caso del autor como en el caso de la directora)
%     \item No entender la metodología de gestión y desarrollo (Kanban)
%     \item Pérdida de información (se solventa con gestión de versiones + github)
%     \item Problemas con la integración de herramientas de desarrollo
%     \item Análisis incorrecto de los usuarios finales del producto (Piensa en el lenguaje, por ejemplo).
%     \item Encontrado un proyecto idéntico (riesgo positivo (o se aceptan o se explotan))
% \end{enumerate}

{
\setlength{\tabcolsep}{2ex} % Table column separation
\renewcommand{\tabularxcolumn}[1]{>{\arraybackslash}m{#1}} % Default column type
\renewcommand\arraystretch{1.2} %
\rowcolors{2}{gray!25}{white} % Alternating row colors
\begin{xltabular}{\textwidth}
    {
    % @{}
    M{0.02\linewidth}
    >{\justifying\arraybackslash}m{0.22\linewidth}
    M{0.17\linewidth}
    M{0.10\linewidth}
    >{\justifying\arraybackslash}m{0.22\linewidth}
    % @{}
    }
    \toprule        % table caption, ref label
        \rowcolor{white!0}
        \textbf{Nº} &
        % \textbf{Tipo} &
        \textbf{Descripción} &
        \textbf{Probabilidad} &
        \textbf{Impacto} &
        % \textbf{Nivel de Riesgo} &
        \textbf{Respuesta} \\ % head first part of table
    \midrule        % line head body
    \endfirsthead   % Definition of 1. table header

    \rowcolor{white!0}
    \multicolumn{5}{c}{Continuación de la página anterior}\\
    \toprule
        \rowcolor{white!0}
        \textbf{Nº} &
        % \textbf{Tipo} &
        \textbf{Descripción} &
        \textbf{Probabilidad} &
        \textbf{Impacto} &
        % \textbf{Nivel de Riesgo} &
        \textbf{Respuesta} \\ % head following parts of table
    \midrule        % line head body
    \endhead        % Delongtab1finition of all following headers

    \midrule
    \rowcolor{white!0}
    \multicolumn{5}{c}{Continua en la siguiente página}\\ % footer 1. (and more) part(s) of table
    \endfoot        % foots of the table without the last one

    \bottomrule
    \rowcolor{white!0}
    \caption{Riesgos del proyecto \label{tab:riesgos-descripcion}}\\
    \endlastfoot    % the last(!!) foot of the table
    % Data:
    1   & Mala gestión del tiempo                                                                                       & Probable              & Alto             & Replanificar + hacer uso de la agilidad de la metodología                                                                                  \\
    2   & No entender la metodología de gestión y desarrollo (Kanban)                                                                                        & Probable              & Alto             & Profundizar en la formación de la metodología y generar cambios en el seguimiento y control                                                \\
    3   & Pérdida de información (se solventa con gestión de versiones + github)                                                                             & Nada probable         & Muy alto         & Hacer uso de control de versiones para guardar todos los datos. En este caso GitHub.                                                       \\
    4   & Problemas de comunicación (por problemas de conexión, enfermedad, etc.)                                       & Poco probable         & Bajo             & Replanificar reuniones?                                                                                                                    \\
    5   & Problemas con la integración de herramientas de desarrollo                                                                                         & Muy probable          & Muy alto         & Buscar alternativas en caso de no poder integrar las herramientas o profundizar en la formación de esta integración para evitar que ocurra \\
    6   & Análisis incorrecto de los usuarios finales del producto (Pensar en el lenguaje, por ejemplo).                                                     & Probable              & Alto             & Reanalizar los perfiles de usuarios objetivos y replantear los productos                                                                   \\
    7   & Encontrado un proyecto idéntico (riesgo positivo (o se aceptan o se explotan))                                                                     & Poco probable         & Medio            & Buscar una forma de hacer uso de éste, ya sea para integrarlo en el proyecto o para usarlo como formación.                                 \\
\end{xltabular}
}

\urgent[inline]{
    Tenemos que diferenciar entre riesgos externos e internos
}

\urgent[inline]{
    Hay riesgos negativos (que afectan malamente al proyecto) y riesgos
    positivos (que afectan positivamente al proyecto)
}

% \subsubsection{Riesgos incurridos no identificados} % (Los riesgos encontrados al
% final del desarrollo)

\subsection{Análisis de viabilidad}

\change[inline]{Viendo los antecedentes, considero que el proyecto es viable}