\section{Metodología}\label{sec:metodologia}

A la hora de gestionar un proyecto, es recomendable definir una metodología de
trabajo, seguimiento y control con el fin de asegurar el correcto cumplimiento y
finalización de su lista de requisitos y objetivos. Tras investigar y plantear
distintas alternativas, se ha escogido la metodología Kanban para desarrollar
este proyecto.

El método Kanban como tal es útil si se desea:
\begin{itemize}
    \item Iniciar un proyecto de manera ágil, rápida, de coste nulo y bajo riesgo. % Getting off to a low-risk, zero-cost, agile, fast start.
    \item Analizar y mejorar el proceso de trabajo ya existente. % Pinning down existing workflows and spotting glaring errors.
    \item Controlar múltiples grupos de tareas. % Controlling multiple pieces of unconnected work.
    \item Asegurar que el número de tareas en ejecución están dentro de un nivel aceptable. % Keeping the numbers of jobs in play down to an acceptable level.
    \item Cambiar a una mentalidad de desarrollo ágil. % Getting the team into an agile way of thinking.
\end{itemize}

Esta sección, por tanto, estará dedicada a explicar los conceptos e ideas
necesarias de cara a justificar la razón por la cual se ha escogido este método
de trabajo. Además, es necesario mencionar que esta información será resumida
principalmente de \cite{Cole2015-fd} y \cite{Stellman2014-qr}, siendo éstas las
referencias recomendadas en caso de desear más detalles al respecto.
\subsection{Definición}

El método Kanban tiene sus orígenes en las fábricas de coches de Toyota a manos
de Taiichi Ohno. Fue creado como un simple sistema de planificación y
administración de trabajo e inventario de cada fase de producción. Sin embargo,
fue David J. Anderson quien definió y adaptó esta metodología para el uso en
ingeniería y desarrollo de software.

Es necesario mencionar que, contrario a lo que se piensa normalmente, Kanban es
una metodología ágil de mejora de procesos y no un framework de administración
de proyectos. Y así como muchos métodos de este tipo, se caracteriza
principalmente por ser evolutivo e iterativo en el tiempo.

Asimismo, Kanban comparte la misma ideología de trabajo con el método Lean hasta
tal punto que se considera que es una especialización de esta última. Aplicando
los principios y valores de este proceso, Kanban se centra en eliminar los
desperdicios de tiempo y recursos que el equipo tiene. Por lo tanto, este
proceso de mejora pide a los equipos de desarrollo empezar con una metodología
ya existente para poder perfeccionarla gradualmente en el tiempo a través de la
experimentación, el cálculo de distintas métricas de rendimiendo y la
confirmación de resultados positivos según dichas medidas.

Todo equipo cuenta con un sistema para la implementación de código, ya sea que
se siga una metodología formal como Scrum o que se disponga de una serie de
reglas no definidas o reconocidas explícitamente. Como consecuencia de esto, lo
único que se necesita para empezar a utilizar Kanban es identificar el proceso
de desarrollo actual para poder formalizarlo y adaptarlo a esta metodología.

Por tanto, aunque Kanban no es un sistema de gestión de proyectos, es posible
hacer uso de éste para ello al tener como principal objetivo aumentar la
predictabilidad del flujo de trabajo y así mejorar la planificación del proyecto
como tal.

\subsection{Principios de Kanban}
Al ser una especialización del método Lean, se tiene una serie de principios
directores en esta metodología. Especificamente se pueden listar los siguientes
tres:
\begin{itemize}
    \item \textbf{Empieza con lo que se hace actualmente}: %Start with what you do now
    Como se había mencionado anteriormente, el método Kanban pide requiere de un
    proceso o metodología inicial. Es a través del análisis y la comprensión de
    dicho proceso que Kanban permite perfeccionarlo iterativamente. Pero como
    consecuencia de esto, no es posible aplicar Kanban adecuadamente si se
    desconoce la metodología que usa el equipo de desarrollo.

    \item \textbf{Comprométete al cambio evolutivo e incremental}: %Agree to pursue incremental, evolutionary change
    Aunque parezca muy repetitivo, es necesario volver a mencionar que el
    objetivo de Kanban es la mejora gradual del sistema. A eso se refiere el
    cambio evolutivo e incremental y es por eso que el método pide calcular
    métricas de control. Es a través de estas medidas que se llegan a conocer
    las partes mejores del proceso de desarrollo.

    \item \textbf{Respeta el proceso, los roles, las responsabilidades y títulos
    actuales}: %Respect the current process, roles, responsibilities and titles
    Cuando el equipo de trabajo dedica tiempo a medir el rendimiento del
    sistema, es posible encontrar ciclos de retroalimentación que contienen
    información importante para la mejora evolutiva del proceso. Sin embargo,
    para poder aplicar dicha información es necesario conocer cómo se aplica
    ésta a cada rol del equipo. Por esta razón Kanban considera relevante
    respetar las responsabilidades y roles asociadas a cada miembro del grupo.
\end{itemize}

\subsection{Prácticas de Kanban}

El método Kanban, además, define explícitamente una serie de prácticas a llevar
a cabo con el fin de aplicar correctamente la mejora de procesos que éste
permite. Sin embargo, no es necesario hacer uso de todas estas en su totalidad
al iniciar con el método. Kanban tiene específicamente los siguientes seis principios:

\begin{itemize}
    \item \textbf{Definir y visualizar el flujo de trabajo}: %Define and Visualise Workflow
    Con el fin de familiarizarse más con el proceso de trabajo, Kanban pide
    hacer uso de representaciones visuales de éste a través de tablones, listas
    de trabajo y elementos de trabajo. La combinación de dichos componentes
    permite definir el flujo de trabajo en su totalidad de una manera fácilmente
    entendible. En Kanban, visualizar significa anotar exactamente lo que hace
    el equipo sin embellecer los detalles con el fin de observar el sistema en
    su totalidad.

    \item \textbf{Limitar el trabajo en progreso}: %Limit work-in-progress
    En el método Kanban los distintos trabajos a realizar en el proceso de
    desarrollo tienen un límite de tareas en ejecución en un determinado
    momento. La justificación que se le da a este principio recae en el hecho de
    que realizar múltiples labores a la vez reduce la eficiencia del equipo. A
    dicho límite se le conoce como \emph{WiP} (de sus siglas en inglés
    \emph{Work-in-Progress}).

    \item \textbf{Manejar el flujo de trabajo}: %Manage the flow of work
    Una vez definido el flujo de trabajo, el objetivo será conseguir una rápida
    y suave transición entre los distintos grupos de tareas, desde la lista de
    trabajo a empezar hasta dar por finalizada la labor. Si se consigue esta
    velocidad de flujo, se dice que se está operando con la eficiencia óptima y
    es en este momento que se crea el máximo valor laboral en el menor tiempo
    posible. A medida que el equipo desarrolla, se van encontrando los cuellos
    de botella y ajustando los límites de trabajo.

    \item \textbf{Hacer explícitas las políticas de proceso}: %Make the process explicit
    Para poder obtener un flujo de trabajo óptimo, además, es necesario definir
    los objetivos que se deben cumplir para dar por terminada una serie de
    tareas con el fin de determinar cuándo se avanza de estado. Dichas
    condiciones de finalización, sin embargo, siempre irán ligadas al tipo de
    proceso que se está utilizando y, por tanto, deberán ser discutidas y
    especificadas por el equipo.

    \item \textbf{Implementar ciclos de retroalimentación}: %Implement Feedback Loops
    Como se había mencionado anteriormente, los ciclos de retroalimentación
    permiten identificar la información necesaria para aplicar mejoras en el
    proceso de desarrollo. Kanban define dichos ciclos como la combinación de
    limitaciones que permiten dar a conocer los puntos débiles del equipo y su
    correcta implementación requiere de experimentación y análisis del
    rendimiento general del sistema.

    \item \textbf{Mejorar colaborativamente}: %Improve collaboratively
    Una vez que Kanban ya está implementado y el foco de atención recae en el
    flujo de trabajo, el carácter de mejora incremental del método vuelve a
    salir a la luz. Es a través de la discusión, análisis y puesta en marcha de
    nuevas ideas que el equipo puede encontrar bloqueos en el proceso y
    adaptarse a cambios rápidamente.
\end{itemize}

\subsection{Métricas de flujo}
\urgent{TRADUCIR Y FINALIZAR}
% Orderly Disruption Limited, Daniel S. Vacanti, Inc. Offered for license under the Attribution
% ShareAlike license of Creative Commons, accessible at http://creativecommons.org/licenses/by-
% sa/4.0/legalcode and also described in summary form at http://creativecommons.org/licenses/by-sa/4.0/,
% By using this Kanban GuideTM, you acknowledge that you have read and agree to be bound by the terms of
% the Attribution ShareAlike license of Creative Commons.

\begin{itemize}
    \item \textbf{Trabajo en progreso}: The number of work items started but not finished
    (according to the DoW).
    \item \textbf{Rendimiento}: The number of work items finished per unit of
    time. Note the measurement of throughput is the exact count of work items.
    \item \textbf{Edad del elemento de trabajo}: The amount of elapsed time between when a work
    item started and the current time.
    \item \textbf{Tiempo de ciclo}: The amount of elapsed time between when a work
    item started and when a work item finished.
\end{itemize}

\subsection{Herramientas de Kanban}
\begin{itemize}
    \item \textbf{Tablón}:
    At the heart of the Kanban method is a deceptively clever tool:
    the Kanban board. Calling these boards a visual to-do list is an
    over-simplification but a decent starting point. The board is a
    graphic representation of the work to be done and the end-to-end
    flow from start to finish. The simplest and some argue the most
    pure Kanban board consists of just three columns: things to do,
    tasks in progress and finally work done. This simple format is
    universal and matches any project or corporate workflow.
    A kanban board is a tool that teams use to visualize their workflow. (The K in the
    methodology name Kanban is typically uppercase; the k in kanban board is usually
    lowercase.) A kanban board looks a lot like a Scrum task board: it typically consists of
    columns drawn on a whiteboard, with sticky notes stuck in each column. (It's more
    common to find sticky notes stuck to kanban boards than it is to find index cards.)
    There are three very important differences between a task board and a kanban board.
    You already learned about the first difference: that kanban boards only have stories,
    and do not show tasks. Another difference is that columns in kanban boards usually
    vary from team to team. Finally, kanban boards can set limits on the amount of work
    in a column. We'll talk about those limits later on; for now, let's concentrate on the
    columns themselves, and how different teams using Kanban will often have different
    columns in their kanban boards. One team's board might have familiar To Do, In Pro-
    gress, and Done columns. But another team's board could have entirely different
    columns.
    When a team wants to adopt Kanban, the first thing that they do is visualize the
    workflow by creating a kanban board. For example, one of the first kanban boards in
    David Anderson's book, Kanban, has these columns: Input Queue, Analysis (In Prog),
    Analysis (Done), Dev Ready, Development (In Prog), Development (Done), Build
    Ready, Test, and Release Ready. This board would be used by a team that follows a
    process where each feature goes through analysis, development, build, and test. So
    they might start off with a kanban board like the one shown in Figure 9-1, with sticky
    notes in the columns representing the work items flowing through the system.
    \item \textbf{Listas}:
    Keep it simple to start with and try out the four previously
    suggested stages: ideas, to do, doing and done. The demarcation between each
    status is clear and in turn generates the triggers for moving cards on:
    \begin{itemize}
        \item \textbf{Backlog}: the maybe or maybe not stage when there's a question
        mark of any sort outstanding. At the very heart of the Kanban board are the to-do items also
        known as the backlog in various agile frameworks. These individual
        items are all delivery focused and must deliver business value
        directly or indirectly. For example, setting up a Kanban board is
        a legitimate item but a meeting to discuss the options is merely
        part of the main job. The tasks are business delivery focused
        and not centred on activities. If an item on the backlog does not
        contribute to business goals, it should be removed.
        \item \textbf{Terminado}: totally complete with nothing more to do except reap
        the rewards or accept the gratitude.
        \item \textbf{Listas intermedias}: To do, In progress, Testing...
    \end{itemize}
    \item \textbf{Cartas}:
\end{itemize}

\subsection{Definiendo los criterios de finalización}

\subsection{Justificación de la metodología}
\change[inline]{Por último, la gestión del proyecto se realizará siguiendo la
metodología de desarrollo ágil Kanban para permitir construir de manera
iterativa las distintas funcionalidades del proyecto. Hacer uso del método
Kanban para el desarrollo y gestión del proyecto.}

\change[inline]{No se puede definir una estimación de tiempo, por lo que se
deberá usar una metodología ágil}

% -----------------------------------------------------------------------------

\subsection{Trello como herramienta de gestión}\label{subsec:trello}

Once the starting format of the Kanban board is agreed, the
first and almost pivotal decision to be made is whether to go for
a physical board or an electronic one. Both have their pros and
cons and there may be working practices that guide the final
decision. A virtual board can't be beaten for accessibility and
ease of sharing, as you're never more than a smart phone or
iPad away. But in our opinion the most important thing is for
the board to be highly visible, and nothing can beat a physical
board for that.
A high-profile, physical board has an almost magical quality, like
a fireplace in a huge front room, and draws people in. To start
with it's more about curiosity, yet after a short while it becomes
a centrepiece and focus for team activities. Work is planned,
prioritised and progressed around the board. A physical board
is also guaranteed to generate huge interest in unlikely places.
Senior management love the visibility of a board, so expect a
visit from the CEO or Finance Director within a week. For
once they'll see what's really going on in the organisation
without quizzing middle management or ploughing through
turgid weekly reports.

Despite our absolute preference for an old-fashioned physical
board, there are times when an electronic board either makes
more sense or is even the only viable option. When individuals
are regularly on the move or if the team is split over different
locations, there are insurmountable physical issues to deal with
and a tech option become more attractive.
But before giving up on having a physical board think carefully,
especially when trialling agile for the first time. A tech alternative
will work well enough from a functional perspective but is far
less visible and engaging, so many soft benefits will be lost.
Don't go down that route just because members of the team
occasionally work from home. Don't throw the baby out with
the bathwater.
If an electronic board is the only workable solution, consider
driving it from a physical source - start with a wall and duplicate.
The overhead of keeping two boards in sync will be offset by the
benefits of having a real board. But when all else fails there are
plenty of electronic options with good coverage across the main
devices. Some are completely free and all the rest offer a trial
period, so try before you buy.

\change[inline]{Uno de los artefactos es el tablón como tal. Aquí se define Trello como tablón online/virtual}
\change[inline]{Definición de Trello y un poco de transfondo}

Se sacará la información de aquí: \cite{Brechner2015-dv} % Teoría aplicada (el de Xbox)
\subsubsection{Significado de las listas}
\change[inline]{Uno de los artefactos son las listas. Aquí se definen las escogidas}

\begin{itemize}
    \item \textbf{Backlog:} Tareas de la iteración en espera de ser empezadas
    \item \textbf{To Do:} Tareas a realizarse
    \item \textbf{Doing:} Tareas que se están realizando
    \item \textbf{Testing:} Tareas que se están evaluando antes de darse por completadas
    \item \textbf{Done:} Tareas de la iteración terminadas
    \item \textbf{Approved:} Tareas discutidas y aprobadas
\end{itemize}

\subsubsection{Formato de las cartas}

\change[inline]{Aquí se definen las cartas que son otro artefacto de Kanban. Conviene mejor usar capturas}

%! Aquí conviene mejor una captura

% \begin{markdown}
%     # Ge.P-1: Definición de la metodología de gestión
%     **Iteraciones:** 2
%     **Código:** FeI
%     **Tarea:** 1
%     **Tiempo Estimado:** 0.0
%     **Tiempo Usado:** 0.0
%     **Desviación:** 0.0
%     **Descripción:** nan
% \end{markdown}

\subsubsection{Significado de etiquetas}
\change[inline]{Un añadido que se puede utilizar son las etiquetas como artefacto. Aquí se definen}
\begin{itemize}
    \item \textbf{Bug:} Cuando se detecta un error y se debe arreglar
    \item \textbf{Bloqueado:} Cuando una tarea no se puede completar debido a
    otras circunstancias
    \item \textbf{Pendiente:} Abreviado de “Pendiente de Retroalimentación”.
    Indica que esta tarea debe discutirse en una reunión
    \item \textbf{No aceptada:} Indica que la tarea no ha sido aceptada para
    continuar en la siguiente iteración y debe retrabajarse, cambiarse o
    descartarse.
\end{itemize}

\subsubsection{Criterios de movimiento de listas}
\change[inline]{Se comentó que hay que definir el concepto de “finalizado” para cada lista. Aquí se hace eso}
% Qué cosas se cumplen para que una carta se mueva a la siguiente lista. Agile Project Managament With Kanban pag. 24 (set limits on chaos and Step 4: Define done)
% Pull Criteria




