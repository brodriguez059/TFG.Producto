\section{Metodología}\label{sec:metodologia}

\change[inline]{Definición de Kanban: “Kanban es un\ldots”}

Se sacará la información de aquí principalmente:
\cite{Cole2015-fd} % Teoría formal
\cite{Stellman2014-qr} % Teoría formal

Kanban's great for:
\begin{itemize}
    \item Getting off to a low-risk, zero-cost, agile, fast start.
    \item Pinning down existing workflows and spotting glaring errors.
    \item Controlling multiple pieces of unconnected work.
    \item Keeping the numbers of jobs in play down to an acceptable level.
    \item Getting the team into an agile way of thinking.
\end{itemize}

However,
there are subtle, yet important, differences with Kanban:

All work is of a similar size. It's better to have smaller
stories of roughly equal size. Splitting large pieces of work
down into smaller, similar sized packages has been proven
to improve workflow and results in more predictive end-to-
end cycle times. Comparing like-for-like can also aid the
review process.

The backlog is refined more regularly. The Kanban
backlog tends to be exceptionally dynamic especially
in a support type environment. Backlogs in other agile
environments are far from static but they're just a touch
less zippy. It's not unusual to review a Kanban backlog on a
daily basis.

Jobs are pulled not pushed. A Kanban team adopts a
what's next policy rather than packaging up work into a
connected delivery. The task assigned the highest priority is
pulled into play when resource becomes available.

\subsection{Principios de Kanban}
\begin{itemize}
    \item \textbf{Empieza con lo que tienes ya}: %Start with what you do now
    The thing about habitual problems is that they're habitual.
When your team does something on a project that will eventually lead to bugs and
missed deadlines, it doesn't feel like a mistake at the time. You can do all the root-
cause analysis that you want after the fact; faced with the same choice, the team will
probably still make the same decision. That's how people are.
For example, let's say that a programming team always finds themselves delivering
software to their users, only to repeatedly have awkward meetings where users can't
find the features they think they were promised. Now, it's certainly possible that these
developers are incredibly forgetful, and that they always forget one or two of the fea‐
tures that they discussed with the users. But it's more likely that they have a recurring
problem with how they gather their requirements or communicate them to the users.
The goal of process improvement is to find recurring problems, figure out what those
problems have in common, and come up with tools to fix them.
The key here is the second part of that sentence: figuring out what those problems
have in common. If you just assume, for example, that a developer simply can't
remember all of the things the users asked for, or that the users constantly change
their minds, then you've effectively decided that the problems are unfixable. But if
you assume that there's a real root cause that's happening over and over again, then
you stand a chance of finding and fixing it.
That's where Kanban starts: taking a look at how you work today, and treating it as a
set of changeable, repeatable steps. Kanban teams call steps or rules that they always
follow policies. This essentially boils down to a team recognizing their habits, seeing
what steps they take every time they build software, and writing all of those things
down.
Writing down the rules that a team follows can sometimes be tricky, because it's easy
to fall into the trap of judging a team—or an individual team member—on the results
of a project: if the project is successful, everyone on the team must be good at their
jobs; if the project fails, they must be incompetent. This is unfair, because it assumes
that everything in the project is within the control of the team. Lean thinking helps
get past this by telling us to see the whole, which in this case means seeing that there's
a bigger system in place.
That's worth repeating: every team has a system for building software. This system may
be chaotic. It may change frequently. It may exist mainly inside the heads of the team
members, who never actually discussed a larger system that they follow. For teams
that follow a methodology like Scrum, the system is codified and understood by
everyone. But many teams follow a system that exists mainly as “tribal” knowledge:
we always start a project by talking to these particular customer representatives, or
building that sort of schedule, or creating story cards, or having programmers jump
in and immediately start to code after a quick meeting with a manager.
    \item \textbf{Comprométete al cambio evolutivo e incremental}: %Agree to pursue incremental, evolutionary change
    This is the system that Kanban starts with. The team already has a way to run their
project. Kanban just asks them to understand that system. That's what it means to
start with what you do now. The goal of Kanban is to make small improvements to
that system. That's what it means to pursue incremental, evolutionary change—and
why Kanban has the practice improve collaboratively, evolve experimentally. In
lean thinking, part of seeing the whole is taking measurements, and measurements
are at the core of experimentation and the scientific method. A Kanban team will
start with their system for building software, and take measurements to get an
objective understanding of it. Then they'll make specific changes as a
team—later in this
chapter, you'll learn exactly how those changes work—and check their measurements
to see if those changes have the desired effect.
    \item \textbf{Respeta el proceso, los roles, las responsabilidades y títulos actuales}: %Respect the current process, roles, responsibilities and titles
    The Lean value of amplifying learning is also an important part of evolving the system that your team uses to build software. Throughout this book you've learned
about feedback loops. When you collaborate to measure the system and evolve
experimentally, those feedback loops become a very important tool for gathering
information and feeding it back into the system; the Kanban practice of
implementing feedback loops should make sense to you, and should help you to see
how Kanban and Lean are closely linked.
Amplifying learning also factors into the Kanban principle of initially
respecting current roles and responsibilities. For example, say that a team
always starts each project
with a meeting between a project manager, a business analyst, and a programmer.
They may not have written down a rule for what goes on in that meeting, but you
probably have a good idea of what goes on in it just from reading those job titles.
That's one reason why Kanban respects current roles, responsibilities, and job titles—
because they're an important part of the system.
A common theme between all of these principles is that Kanban only works for a
team when they take the time to understand their own system for building software.
If there was one right way to build software, everyone would just use it. But we
started this book by saying back in Chapter 2 that there is no silver bullet—there's no
single set of “best” practices that will guarantee that a team builds software right every
time. Even the same team, using the same practices, can have success with one project
but fail miserably in the next one. This is why Kanban starts with understanding the
current system for running the project: once you see the whole system, Kanban gives
you practices to improve it.
\end{itemize}

\subsection{Prácticas de Kanban}

% It is not expected that implementations adopt all six practices initially. Partial imple‐
% mentations are referred to as “shallow” with the expectation that they gradually
% increase in depth as more practices are adopted and implemented with greater
% capability.

\begin{itemize}
    \item \textbf{Definir y visualizar el flujo de trabajo}: %Define and Visualise Workflow
    \item \textbf{Limitar el trabajo en progreso}: %Limit work-in-progress
    \item \textbf{Manejar el flujo de trabajo}: %Manage the flow of work
    \item \textbf{Hacer explícitas las políticas de proceso}: %Make the process explicit
    \item \textbf{Implementar ciclos de retroalimentación}: %Implement Feedback Loops
    \item \textbf{Mejorar colaborativamente}: %Improve collaboratively
\end{itemize}

\subsection{Métricas de flujo}
\urgent{TRADUCIR Y FINALIZAR}
% Orderly Disruption Limited, Daniel S. Vacanti, Inc. Offered for license under the Attribution
% ShareAlike license of Creative Commons, accessible at http://creativecommons.org/licenses/by-
% sa/4.0/legalcode and also described in summary form at http://creativecommons.org/licenses/by-sa/4.0/,
% By using this Kanban GuideTM, you acknowledge that you have read and agree to be bound by the terms of
% the Attribution ShareAlike license of Creative Commons.

\begin{itemize}
    \item \textbf{Trabajo en progreso}: The number of work items started but not finished
    (according to the DoW).
    \item \textbf{Rendimiento}: The number of work items finished per unit of
    time. Note the measurement of throughput is the exact count of work items.
    \item \textbf{Edad del elemento de trabajo}: The amount of elapsed time between when a work
    item started and the current time.
    \item \textbf{Tiempo de ciclo}: The amount of elapsed time between when a work
    item started and when a work item finished.
\end{itemize}

\subsection{Herramientas de Kanban}
\begin{itemize}
    \item \textbf{Tablón}:
    At the heart of the Kanban method is a deceptively clever tool:
    the Kanban board. Calling these boards a visual to-do list is an
    over-simplification but a decent starting point. The board is a
    graphic representation of the work to be done and the end-to-end
    flow from start to finish. The simplest and some argue the most
    pure Kanban board consists of just three columns: things to do,
    tasks in progress and finally work done. This simple format is
    universal and matches any project or corporate workflow.
    \item \textbf{Listas}:
    Keep it simple to start with and try out the four previously
    suggested stages: ideas, to do, doing and done. The demarcation between each
    status is clear and in turn generates the triggers for moving cards on:
    \begin{itemize}
        \item \textbf{Backlog}: the maybe or maybe not stage when there's a question
        mark of any sort outstanding. At the very heart of the Kanban board are the to-do items also
        known as the backlog in various agile frameworks. These individual
        items are all delivery focused and must deliver business value
        directly or indirectly. For example, setting up a Kanban board is
        a legitimate item but a meeting to discuss the options is merely
        part of the main job. The tasks are business delivery focused
        and not centred on activities. If an item on the backlog does not
        contribute to business goals, it should be removed.
        \item \textbf{Terminado}: totally complete with nothing more to do except reap
        the rewards or accept the gratitude.
        \item \textbf{Listas intermedias}: To do, In progress, Testing...
    \end{itemize}
    \item \textbf{Cartas}:
\end{itemize}

\subsection{Definiendo los criterios de finalización}

\subsection{Justificación de la metodología}
\change[inline]{Por último, la gestión del proyecto se realizará siguiendo la
metodología de desarrollo ágil Kanban para permitir construir de manera
iterativa las distintas funcionalidades del proyecto. Hacer uso del método
Kanban para el desarrollo y gestión del proyecto.}

\change[inline]{No se puede definir una estimación de tiempo, por lo que se
deberá usar una metodología ágil}

% -----------------------------------------------------------------------------

\subsection{Trello como herramienta de gestión}\label{subsec:trello}

Once the starting format of the Kanban board is agreed, the
first and almost pivotal decision to be made is whether to go for
a physical board or an electronic one. Both have their pros and
cons and there may be working practices that guide the final
decision. A virtual board can't be beaten for accessibility and
ease of sharing, as you're never more than a smart phone or
iPad away. But in our opinion the most important thing is for
the board to be highly visible, and nothing can beat a physical
board for that.
A high-profile, physical board has an almost magical quality, like
a fireplace in a huge front room, and draws people in. To start
with it's more about curiosity, yet after a short while it becomes
a centrepiece and focus for team activities. Work is planned,
prioritised and progressed around the board. A physical board
is also guaranteed to generate huge interest in unlikely places.
Senior management love the visibility of a board, so expect a
visit from the CEO or Finance Director within a week. For
once they'll see what's really going on in the organisation
without quizzing middle management or ploughing through
turgid weekly reports.

Despite our absolute preference for an old-fashioned physical
board, there are times when an electronic board either makes
more sense or is even the only viable option. When individuals
are regularly on the move or if the team is split over different
locations, there are insurmountable physical issues to deal with
and a tech option become more attractive.
But before giving up on having a physical board think carefully,
especially when trialling agile for the first time. A tech alternative
will work well enough from a functional perspective but is far
less visible and engaging, so many soft benefits will be lost.
Don't go down that route just because members of the team
occasionally work from home. Don't throw the baby out with
the bathwater.
If an electronic board is the only workable solution, consider
driving it from a physical source - start with a wall and duplicate.
The overhead of keeping two boards in sync will be offset by the
benefits of having a real board. But when all else fails there are
plenty of electronic options with good coverage across the main
devices. Some are completely free and all the rest offer a trial
period, so try before you buy.

\change[inline]{Uno de los artefactos es el tablón como tal. Aquí se define Trello como tablón online/virtual}
\change[inline]{Definición de Trello y un poco de transfondo}

Se sacará la información de aquí: \cite{Brechner2015-dv} % Teoría aplicada (el de Xbox)
\subsubsection{Significado de las listas}
\change[inline]{Uno de los artefactos son las listas. Aquí se definen las escogidas}

\begin{itemize}
    \item \textbf{Backlog:} Tareas de la iteración en espera de ser empezadas
    \item \textbf{To Do:} Tareas a realizarse
    \item \textbf{Doing:} Tareas que se están realizando
    \item \textbf{Testing:} Tareas que se están evaluando antes de darse por completadas
    \item \textbf{Done:} Tareas de la iteración terminadas
    \item \textbf{Approved:} Tareas discutidas y aprobadas
\end{itemize}

\subsubsection{Formato de las cartas}

\change[inline]{Aquí se definen las cartas que son otro artefacto de Kanban. Conviene mejor usar capturas}

%! Aquí conviene mejor una captura

% \begin{markdown}
%     # Ge.P-1: Definición de la metodología de gestión
%     **Iteraciones:** 2
%     **Código:** FeI
%     **Tarea:** 1
%     **Tiempo Estimado:** 0.0
%     **Tiempo Usado:** 0.0
%     **Desviación:** 0.0
%     **Descripción:** nan
% \end{markdown}

\subsubsection{Significado de etiquetas}
\change[inline]{Un añadido que se puede utilizar son las etiquetas como artefacto. Aquí se definen}
\begin{itemize}
    \item \textbf{Bug:} Cuando se detecta un error y se debe arreglar
    \item \textbf{Bloqueado:} Cuando una tarea no se puede completar debido a
    otras circunstancias
    \item \textbf{Pendiente:} Abreviado de “Pendiente de Retroalimentación”.
    Indica que esta tarea debe discutirse en una reunión
    \item \textbf{No aceptada:} Indica que la tarea no ha sido aceptada para
    continuar en la siguiente iteración y debe retrabajarse, cambiarse o
    descartarse.
\end{itemize}

\subsubsection{Criterios de movimiento de listas}
\change[inline]{Se comentó que hay que definir el concepto de “finalizado” para cada lista. Aquí se hace eso}
% Qué cosas se cumplen para que una carta se mueva a la siguiente lista. Agile Project Managament With Kanban pag. 24 (set limits on chaos and Step 4: Define done)
% Pull Criteria




