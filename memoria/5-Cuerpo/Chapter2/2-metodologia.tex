\section{Metodología}\label{sec:metodologia}
\urgent[inline]{Explicar Kanban}
Por último, la gestión del proyecto se realizará siguiendo la metodología de desarrollo ágil Kanban para permitir construir de manera iterativa las distintas funcionalidades del proyecto.
Hacer uso del método Kanban para el desarrollo y gestión del proyecto.

\urgent[inline]{Desglosar más lo que es Kanban}


% -----------------------------------------------------------------------------


\subsection{Trello como herramienta de gestión}\label{subsec:trello}
\subsubsection{Significado de las listas}
\begin{itemize}
    \item \textbf{Backlog:} Tareas de la iteración en espera de ser empezadas
    \item \textbf{To Do:} Tareas a realizarse
    \item \textbf{Doing:} Tareas que se están realizando
    \item \textbf{Testing:}
    \item \textbf{Done:} Tareas de la iteración terminadas
    \item \textbf{Approved:} Tareas discutidas y aprobadas
\end{itemize}

\subsubsection{Significado de etiquetas}
\begin{itemize}
    \item \textbf{Bug:}
    \item \textbf{Bloqueado:}
    \item \textbf{Pendiente:}
    \item \textbf{No aceptada:}
\end{itemize}

\subsubsection{Formato de las cartas}
