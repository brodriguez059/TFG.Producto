\section{Metodología}\label{sec:metodologia}

\change[inline]{Definición de Kanban: “Kanban es un\ldots”}

Se sacará la información de aquí principalmente:
\cite{Cole2015-fd} % Teoría formal
\cite{Stellman2014-qr} % Teoría formal

\subsection{Principios de Kanban}
\begin{itemize}
    \item \textbf{Empieza con lo que tienes ya}: %Start with what you do now
    \item \textbf{Comprométete al cambio evolutivo e incremental}: %Agree to pursue incremental, evolutionary change
    \item \textbf{Respeta el proceso, los roles, las responsabilidades y títulos actuales}: %Respect the current process, roles, responsibilities and titles
\end{itemize}

\subsection{Prácticas de Kanban}
\begin{itemize}
    \item \textbf{Definir y visualizar el flujo de trabajo:} %Define and Visualise Workflow
    \item \textbf{Limitar el trabajo en progreso:} %Limit work-in-progress
    \item \textbf{Manejar el flujo de trabajo:} %Manage the flow of work
    \item \textbf{Hacer explícito el proceso:} %Make the process explicit
    \item \textbf{Mejorar colaborativamente:} %Improve collaboratively
\end{itemize}

% \subsection{Métricas de flujo}
%! Cuidado, esta licencia es problemática y podría resultar complicado cumplir con ella
% Orderly Disruption Limited, Daniel S. Vacanti, Inc. Offered for license under the Attribution
% ShareAlike license of Creative Commons, accessible at http://creativecommons.org/licenses/by-
% sa/4.0/legalcode and also described in summary form at http://creativecommons.org/licenses/by-sa/4.0/,
% By using this Kanban GuideTM, you acknowledge that you have read and agree to be bound by the terms of
% the Attribution ShareAlike license of Creative Commons.

% \begin{itemize}
%     \item \textbf{WIP}: The number of work items started but not finished
%     (according to the DoW).
%     \item \textbf{Throughput}: The number of work items finished per unit of
%     time. Note the measurement of throughput is the exact count of work items.
%     \item \textbf{Work Item Age}: The amount of elapsed time between when a work
%     item started and the current time.
%     \item \textbf{Cycle Time}: The amount of elapsed time between when a work
%     item started and when a work item finished.
% \end{itemize}

\subsection{Artefactos de Kanban}
\begin{itemize}
    \item \textbf{Tablón}:
    \item \textbf{Listas}:
    \item \textbf{Cartas}:
\end{itemize}

\subsection{Definiendo los criterios de finalización}

\subsection{Justificación de la metodología}
\change[inline]{Por último, la gestión del proyecto se realizará siguiendo la
metodología de desarrollo ágil Kanban para permitir construir de manera
iterativa las distintas funcionalidades del proyecto. Hacer uso del método
Kanban para el desarrollo y gestión del proyecto.}

\change[inline]{No se puede definir una estimación de tiempo, por lo que se
deberá usar una metodología ágil}

% -----------------------------------------------------------------------------

\subsection{Trello como herramienta de gestión}\label{subsec:trello}

\change[inline]{Uno de los artefactos es el tablón como tal. Aquí se define Trello como tablón online/virtual}
\change[inline]{Definición de Trello y un poco de transfondo}

Se sacará la información de aquí:
\cite{Brechner2015-dv} % Teoría aplicada (el de Xbox)
\subsubsection{Significado de las listas}
\change[inline]{Uno de los artefactos son las listas. Aquí se definen las escogidas}

\begin{itemize}
    \item \textbf{Backlog:} Tareas de la iteración en espera de ser empezadas
    \item \textbf{To Do:} Tareas a realizarse
    \item \textbf{Doing:} Tareas que se están realizando
    \item \textbf{Testing:} Tareas que se están evaluando antes de darse por completadas
    \item \textbf{Done:} Tareas de la iteración terminadas
    \item \textbf{Approved:} Tareas discutidas y aprobadas
\end{itemize}

\subsubsection{Formato de las cartas}

\change[inline]{Aquí se definen las cartas que son otro artefacto de Kanban. Conviene mejor usar capturas}

%! Aquí conviene mejor una captura

% \begin{markdown}
%     # Ge.P-1: Definición de la metodología de gestión
%     **Iteraciones:** 2
%     **Código:** FeI
%     **Tarea:** 1
%     **Tiempo Estimado:** 0.0
%     **Tiempo Usado:** 0.0
%     **Desviación:** 0.0
%     **Descripción:** nan
% \end{markdown}

\subsubsection{Significado de etiquetas}
\change[inline]{Un añadido que se puede utilizar son las etiquetas como artefacto. Aquí se definen}
\begin{itemize}
    \item \textbf{Bug:} Cuando se detecta un error y se debe arreglar
    \item \textbf{Bloqueado:} Cuando una tarea no se puede completar debido a
    otras circunstancias
    \item \textbf{Pendiente:} Abreviado de “Pendiente de Retroalimentación”.
    Indica que esta tarea debe discutirse en una reunión
    \item \textbf{No aceptada:} Indica que la tarea no ha sido aceptada para
    continuar en la siguiente iteración y debe retrabajarse, cambiarse o
    descartarse.
\end{itemize}

\subsubsection{Criterios de movimiento de listas}
\change[inline]{Se comentó que hay que definir el concepto de “finalizado” para cada lista. Aquí se hace eso}
% Qué cosas se cumplen para que una carta se mueva a la siguiente lista. Agile Project Managament With Kanban pag. 24 (set limits on chaos and Step 4: Define done)
% Pull Criteria




