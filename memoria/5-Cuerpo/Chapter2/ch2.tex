\chapter{Simulación}\label{ch:simulacion}

% Bloque simulación de sistemas

\change[inline]{Definición de simulación}
\change[inline]{Explicación sobre simulación de sistemas.}
\change[inline]{¿Por qué y para qué la simulación de sistemas?}
\change[inline]{Puedes citar a uno de tus libros aquí.}

\section{Conceptos básicos}

\subsection{Modelo}
\change[inline]{Haz una cita a la definición de tu profesor de simulación de sistemas (igual no se puede)} %https://normas-apa.org/referencias/citar-curso-o-material-de-clase/
\change[inline]{Definición de modelo (Nos centraremos en modelos probabilísticos)}
\change[inline]{¿Clasificación de modelos? (los modelos de simulación deben ser simbólicos (no tienen una relación física o analógica con el sistema real, sino una relación lógica))}
\change[inline]{¿Qué significa ser dinámico?}
\change[inline]{¿Qué significa ser discreto?}

\subsection{Sistema}
\change[inline]{Definición de sistema}
\change[inline]{Partes y conceptos de un sistema de simulación (entorno del sistema, entidad, atributo, actividad, estado, suceso, término endógeno, término exógeno, contadores estadísticos, medidas de rendimiento)}

\section{Ventajas e inconvenientes}
\change[inline]{Ventajas e inconvenientes}

\section{Diseño basado en eventos}
\change[inline]{Grafo de sucesos}

\section{Sistemas dinámicos discretos}
\change[inline]{Partes fundamentales (en común) de los simuladores dinámicos discretos (lista de sucesos, temporizador de eventos, reloj...)}




