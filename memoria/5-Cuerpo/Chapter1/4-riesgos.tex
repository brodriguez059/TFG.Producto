\section{Análisis de riesgos y viabilidad}

\subsection{Análisis de riesgos}

\subsubsection{Métricas}

% PMBOK - Project Management Body of Knowledge 6º edición

\begin{table}[H]
    \centering
    \begin{tabular}{cc|cc|cc}
    \textbf{Probabilidad} & \textbf{Valor} & \textbf{Impacto} & \textbf{Valor} & \textbf{Tipo de Riesgo} & \textbf{Probabilidad x Impacto} \\
    Nada probable         & 0.10           & Muy bajo         & 0.05           & Muy bajo                & De 0 a 0.04                     \\
    Poco probable         & 0.30           & Bajo             & 0.10           & Bajo                    & De 0.05 a 0.09                  \\
    Probable              & 0.50           & Moderado         & 0.20           & Moderado                & De 0.10 a 0.29                  \\
    Muy probable          & 0.70           & Alto             & 0.40           & Alto                    & De 0.30 a 0.49                  \\
    Casi probable         & 0.90           & Muy alto         & 0.80           & Muy alto                & Mayor a 0.50                    \\
    \end{tabular}
    \caption{Métricas para el análisis de riesgos}
    \label{tab:riesgos-metricas}
\end{table}

\subsubsection{Tabla de riesgos}
% \begin{enumerate}
%     \item Mala gestión del tiempo (falta de tiempo + pérdida de tiempo)
%     \item Problemas de contacto con [directora] (por problemas de conexión,
%     enfermedad, etc. Tanto en el caso del autor como en el caso de la directora)
%     \item No entender la metodología de gestión y desarrollo (Kanban)
%     \item Pérdida de información (se solventa con gestión de versiones + github)
%     \item Problemas con la integración de herramientas de desarrollo
%     \item Análisis incorrecto de los usuarios finales del producto (Piensa en el lenguaje, por ejemplo).
%     \item Encontrado un proyecto idéntico (riesgo positivo (o se aceptan o se explotan))
% \end{enumerate}

\begin{table}[H]
    \centering
    \resizebox{\textwidth}{!}{%
    \begin{tabular}{l|l|l|l|l|l|l}
    \textbf{Nº} & \textbf{Tipo} & \textbf{Descripción}                                                                                                                               & \textbf{Probabilidad} & \textbf{Impacto} & \textbf{Nivel de Riesgo} & \textbf{Respuesta}                                                                                                                         \\
    1           & I             & Mala gestión del tiempo (falta de tiempo + pérdida de tiempo)                                                                                      & Probable              & Alto             & Moderado                 & Replanificar + hacer uso de la agilidad de la metodología                                                                                  \\
    2           & E             & Problemas de contacto con {[}directora{]} (por problemas de conexión, enfermedad, etc. Tanto en el caso del autor como en el caso de la directora) & Poco probable         & Bajo             & Muy bajo                 & Replanificar reuniones?                                                                                                                    \\
    3           & I             & No entender la metodología de gestión y desarrollo (Kanban)                                                                                        & Probable              & Alto             & Moderado                 & Profundizar en la formación de la metodología y generar cambios en el seguimiento y control                                                \\
    4           & E             & Pérdida de información (se solventa con gestión de versiones + github)                                                                             & Nada probable         & Muy alto         & Bajo                     & Hacer uso de control de versiones para guardar todos los datos. En este caso GitHub.                                                       \\
    5           & I             & Problemas con la integración de herramientas de desarrollo                                                                                         & Muy probable          & Muy alto         & Muy alto                 & Buscar alternativas en caso de no poder integrar las herramientas o profundizar en la formación de esta integración para evitar que ocurra \\
    6           & I             & Análisis incorrecto de los usuarios finales del producto (Piensa en el lenguaje, por ejemplo).                                                     & Probable              & Alto             & Moderado                 & Reanalizar los perfiles de usuarios objetivos y replantear los productos                                                                   \\
    7           & I             & Encontrado un proyecto idéntico (riesgo positivo (o se aceptan o se explotan))                                                                     & Poco probable         & Moderado         & Bajo                     & Buscar una forma de hacer uso de éste, ya sea para integrarlo en el proyecto o para usarlo como formación.                                 \\
    \end{tabular}%
    }
    \caption{Riesgos del proyecto}
    \label{tab:riesgos-descripcion}
    \end{table}

\urgent[inline]{
    Tenemos que diferenciar entre riesgos externos e internos
}

\urgent[inline]{
    Hay riesgos negativos (que afectan malamente al proyecto) y riesgos
    positivos (que afectan positivamente al proyecto)
}

% \subsubsection{Riesgos incurridos no identificados} % (Los riesgos encontrados al
% final del desarrollo)

\subsection{Análisis de viabilidad}

Considero que el proyecto es viable