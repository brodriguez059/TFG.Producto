\section{Tareas y estimación de dedicaciones}\label{sec:tareas}
\change[inline]{No se puede predecir porque usamos Kanban. Sólo conocemos las fechas finales de
entrega.}



% -----------------------------------------------------------------------------


\subsection{Descripción de tareas a realizar}\label{subsec:tareas-descripcion}

\subsubsection{Gestión}
\paragraph{Planificación}
\begin{enumerate}
    \item Definición de la metodología de gestión.
    \item Generación de la planificación inicial.
    \item Automatización de herramientas de gestión con Trello.
    \item Cambios y actualizaciones de la planificación.
\end{enumerate}
\paragraph{Seguimiento y Control}
\begin{enumerate}
    \item Recogida de información sobre el desarrollo del proyecto
    \item Contraste de la información de seguimiento con los planes, identificación de desviaciones significativas y actuación ante riesgos emergentes.
    % \item Aseguramiento de las condiciones para el éxito del proyecto.
    \item Preparación de documentos de cara a la próxima reunión.
    \item Reuniones de fin de iteración
\end{enumerate}

\subsubsection{Formación e Investigación}
\begin{enumerate}
    \item Investigación y formación sobre la metodología Kanban.
    \item Profundización sobre conceptos de simulación de sistemas.
    \item Profundización sobre conceptos de compilación.
    \item Investigación y formación sobre herramientas de desarrollo de CLIs.
    \item Investigación y formación sobre generación de paquetes instalables
    para Python.
    \item Investigación y formación sobre funciones de alto nivel y paquetes
    relacionados de Python.
    \item Investigación sobre lenguajes de modelado y simulación.
    \item Investigación sobre antecedentes del proyecto.
    \item Exploración de alternativas al módulo de lenguaje.
\end{enumerate}

\subsubsection{Producto}
\paragraph{Memoria}
\begin{enumerate}
    \item Diseño y preparación de la estructura
    % \item Configuración del preámbulo %(Igual va en IyP)
    \item Redacción del resumen %(Resumen ejecutivo junto a las palabras clave)
    \item Redacción de la introducción %(Vamos a poner aquí como mucho dos página)
    \item Redactar contexto %(Aquí explicamos más a fondo el problema)
    \item Redacción de la gestión del proyecto %(Generamos la planificación y luego ponemos el seguimiento aquí mismo)
    \item Redacción del marco teórico %(Conceptos teóricos necesarios para entender el proyecto)
    \item Redacción del desarrollo %(El propio desarrollo del producto principal)
    % \item Redacción del apartado de pruebas %(Aquí pondré las pruebas que haré)
    \item Redacción de conclusiones %(Qué puedo sacar de esto, qué trabajo futuro surge de esto)
    \item Preparación e inclusión de referencias bibliográficas %(Cuidado con bibtex)
    \item Redacción de anexos %(Aquí cualquier cosa que desee añadir, actas...)
    \item Validación y corrección de los índices %(A veces fallan estos)
    \item Maquetación y diseño de la memoria %(Esto es literalmente la configuración del preámbulo)
\end{enumerate}
\paragraph{Framework}
\begin{itemize}
    \item \textbf{Lenguaje:}
    \begin{enumerate}
        \item Diseño del léxico del lenguaje.
        \item Diseño de la sintaxis del lenguaje.
        \item Diseño de la semántica del lenguaje.
        \item Diseño de módulos de cara a la implementación del lenguaje.
        \item Implementación del analizador léxico.
        \item Implementación del analizador sintáctico.
        \item Implementación de la tabla de símbolos.
        \item Implementación del análisis semántico.
        \item Documentación de los ficheros del lenguaje.
        \item Pruebas y verificación del lenguaje.
    \end{enumerate}
    \item \textbf{Núcleo:}
    \begin{enumerate}
        \item Diseño de los módulos y paquetes del núcleo.
        % Cosas en común entre todos los modelos de simulación:
        \item Implementación de la estructura de datos para almacenar eventos.
        \item Implementación del reloj y el proceso de temporización de sucesos.
        \item Implementaciones de los procesos de inicialización y finalización.
        \item Implementación del generador de informes final.
        \item Implementación de distintos generadores de datos aleatorios.
        % Cosas variables para cada modelo de simulación:
        \item Implementación de la inicialización de las variables globales,
        variables de entrada, contadores estadísticos y medidas de rendimiento
        para cualquier modelo.
        \item Implementación de la inclusión de nuevos tipos de eventos y sus
        funciones de ejecución correspondientes.
        \item Implementación de distintas estrategias de generación y
        eliminación de eventos dentro de la lista de sucesos.
        \item Implementación de la especificación de una función de parada
        definida por el usuario.
        % Funcionalidad de calidad:
        \item Implementación de la funcionalidad de configuración de la simulación.
        \item Documentación de los ficheros del núcleo.
        \item Pruebas y verificación del núcleo.
    \end{enumerate}
    \item \textbf{CLI:}
    \begin{enumerate}
        \item Diseño de las funcionalidades de la CLI.
        \item Implementación del generador de proyectos.
        \item Implementación del gestor de configuraciones del proyecto.
        \item Implementación del lanzador de modelos de simulación.
        \item Documentación de los ficheros de la CLI.
        \item Pruebas y verificación de la CLI.
    \end{enumerate}
\end{itemize}

\paragraph{Manual}
\begin{enumerate}
    \item Diseño y preparación de la estructura
    % \item Configuración del preámbulo %(Igual va en IyP)
    \item Redacción de la introducción al producto
    % \item Quickstart/Overview %(No sé si esto irá en la introducción)
    \item Especificación de dependencias y requisitos
    \item Redacción del proceso de instalación

    \item Redacción de la explicación del módulo del lenguaje
    \item Redacción de la explicación del núcleo del framework
    \item Redacción de la explicación de la CLI
    % \item Explicación de la creación, configuración y ejecución de modelos
    % (Esto es lo de la CLI)

    \item Generación de ejemplos de uso %(Necesitamos unos ejemplos prácticos, ¿no?)

    \item Preparación e inclusión de referencias bibliográficas %(Cuidado con bibtex)
    \item Redacción de anexos %(Aquí cualquier cosa que desee añadir, especificación del lenguaje, ETDS?...)
    \item Validación y corrección de los índices %(A veces fallan estos)
    \item Maquetación y diseño del manual %(Esto es literalmente la configuración del preámbulo)

\end{enumerate}

\subsubsection{Instalación y Paquetes}
\begin{enumerate}
    \item Preparación de paquetes para LaTeX.
    \item Preparación del entorno de trabajo.
    \item Instalación de paquetes para el desarrollo del lenguaje.
    \item Instalación de paquetes para el desarrollo del núcleo.
    \item Instalación de paquetes para el desarrollo de la CLI.
\end{enumerate}


% -----------------------------------------------------------------------------


% \subsection{Dependencias entre tareas}


% -----------------------------------------------------------------------------


% \subsection{Periodo de desarrollo de tareas}


% -----------------------------------------------------------------------------


\subsection{Estimación de dedicación}\label{subsec:tareas-estimacion}

\begin{table}[H]
    \centering
    \begin{tabular}{ccc}
    \textbf{Paquete} & \textbf{Nombre}                 & \textbf{Estimación (Horas)} \\
    Ge.P             & Gestión - Planificación         & 25                          \\
    Ge.S             & Gestión - Seguimiento           & 20                          \\
    FeI              & Formación e Investigación       & 50                          \\
    Pr.Me            & Proyecto - Memoria              & 50                          \\
    Pr.Fr.Le         & Proyecto - Framework - Lenguaje & 48                          \\
    Pr.Fr.Co         & Proyecto - Framework - Core     & 52                          \\
    Pr.Fr.Cli        & Proyecto - Framework - CLI      & 25                          \\
    Pr.Ma            & Proyecto - Manual               & 15                          \\
    IyP              & Instalaciones y Paquetes        & 15                          \\
    \end{tabular}
    \caption{Estimación de tiempos de dedicación por paquetes de trabajo}
    \label{tab:estimacion-paquetes}
\end{table}


% -----------------------------------------------------------------------------


% \subsection{Hitos de desarrollado}
