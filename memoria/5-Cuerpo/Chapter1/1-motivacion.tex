\section{Motivación}\label{sec:motivacion}

% Bloque simulación de sistemas
% \change[inline]{En simulación de sistemas, Un tipo de modelo son los modelos
% dinámicos discretos.}
% \change[inline]{Explicación de lo que son básicamente. (Igual una referencia a
% la definición de "dinámicos" y a la de "discretos")}
% \change[inline]{Estos tipos de modelos se suelen modelar usando diseños basados
% en eventos a través de grafos de sucesos.}
% \change[inline]{Estos modelos se suelen simular con un diseño basado en eventos
% o sucesos. Que vendría a ser...}
En el área de simulación de sistemas, una familia de modelos son los “modelos
dinámicos discretos”, siendo estos “dinámicos” porque su comportamiento cambia
en el tiempo y “discretos” porque estos cambios ocurren en instantes
determinados en vez de ocurrir continuamente.\improvement{Siento que estás
repitiéndote mucho, tienes que cambiar la calidad de este discurso}
Dichos sistemas se suelen modelar usando un diseño “basado en eventos”, el cual
nos indica que los cambios del modelo se deben considerar como “eventos” que
pueden ocurrir y dar lugar a otros eventos. Para ello, es común generar un
“grafo de sucesos” que represente todos los eventos posibles que pueden ocurrir
en el sistema y las relaciones que tienen entre estos.

% \change[inline]{A pesar de que cada modelo de este tipo tiene sus
% especificidades, todos estos comparten elementos en común independientes del
% sistema simulado (reloj, temporizador de eventos, lista de sucesos, etc.).}
% \change[inline]{Para poder implementar cada modelo, por tanto, es necesario
% añadir dichas implementaciones aparte de las propias del modelo. Sin embargo,
% sabiendo que todas éstas son compartidas, no se le debería dar al programador la
% responsabilidad de hacerlo si se pueden generar ya de antemano.}
Podríamos decir que cada modelo tendrá sus peculiaridades y diferencias
específicas a la hora de generar su implementación. Sin embargo, se da el hecho
de que todos los modelos de esta familia comparten elementos en común
independientemente del sistema a simular: el reloj de la simulación, el
temporizador de eventos, la lista de sucesos, entre otros. Por tanto, podemos
abstraer el desarrollo de estos programas de tal forma que el desarrollador sólo
deba encargarse de implementar todo aquello que sea único del modelo, quitándole
así la responsabilidad de generar los elementos en común y agilizando el
desarrollo en el proceso.




% Bloque de compilación
% \change[inline]{Los compiladores son usados...}
% \change[inline]{Los diseños basados en eventos en realidad cumplen con esta
% peculiaridad}
% \change[inline]{Podemos generar un analizador léxico, sintáctico y semántico
% que nos permita escribir de manera más eficaz nuestro modelo para luego
% convertirlo a algo ejecutable.}
% \change[inline]{Nosotros lo transformaremos a código Python, por lo tanto
% crearemos un transpilador.}
Tomando en cuenta que la principal herramienta de diseño de estos modelos serán
los grafos de sucesos, nos encontraremos con el hecho de que éstos siguen una
estructura representable a través de una gramática independiente de contexto.
Por tanto, es posible generar una serie de analizadores léxico, sintáctico y
semánticos que nos permitan procesar un lenguaje formal a otro código fuente.
Por esta razón hemos considerado una buena opción el uso de herramientas de
generación de compiladores como lo son Flex y Bison. Vemos que es posible crear
estos analizadores de forma que este hipotético nuevo lenguaje sea traducido a
código Python.

% \urgent{Generar una mejor explicación del problema encontrado. Hay que vender este TFG}

% El desarrollo de simuladores de sistemas dinámicos discretos, a pesar de que
% puede realizarse a mano, es capaz de llegar a resultar tedioso cuando se deben
% construir múltiples modelos distintos. Sin embargo, casi siempre se tendrá que
% estos sistemas harán uso de varias implementaciones que no dependen del
% simulador como tal.

% Como todas estos programas compartirán una serie de elementos en común (lista
% de eventos, reloj de la simulación, temporizador del simulador, entre otros)
% independientemente del modelo implementado, se plantea crear un nuevo lenguaje
% de programación orientado al desarrollo rápido de sistemas de simulación
% dinámicos discretos, creando para ello también un transpilador que traduzca
% estas implementaciones a código Python.