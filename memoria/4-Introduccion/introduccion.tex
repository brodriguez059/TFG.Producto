\chapter{Introducción}\label{ch:introduccion}

% Bloque simulación de sistemas \change[inline]{Deberías hablar aquí de la
% simulación de sistemas}
El modelado matemático o modelado analítico se aprovecha de las características
del problema cuya respuesta se desea obtener para llegar a la mejor conclusión.
Sin embargo, muchas veces nos encontraremos con problemas que no se pueden
modelar analíticamente debido a su complejidad en tiempo o espacio. En estas
situaciones, debemos hacer uso de técnicas de aproximación de resultados para
dar con soluciones que, aunque no se pueden garantizar óptimas, sí que se pueden
considerar lo suficientemente buenas. Una de estas técnicas vendría a ser la
simulación de sistemas, la cual a través de modelado simbólico/lógico intenta
reproducir el comportamiento de un determinado sistema con el fin de analizar
una serie de resultados a escoger.

% \change[inline]{¿Qué podría decir que es un modelo de simulación brevemente?
% (Menciona que más adelante explicarás más al respecto)} \change[inline]{¿Por
% qué y para qué la simulación de sistemas?} % Igual esto se va a contexto
Nosotros hablaremos de una definición más exacta de “Simulación de Sistemas” y
“Modelado” más adelante, pero por ahora podemos citar la definición encontrada
en %\cite{banks2010discrete-event}:

\begin{quote}
    \emph{Una simulación es la imitación del comportamiento de un proceso o
    sistema del mundo real en el tiempo.}\textcolor{red}{Simulation involves the
    generation of an artificial history of the system, and the observation of
    that artificial history to draw inferences concerning the operating
    characteristics of the real system that is represented.}
\end{quote}
\change{Toca traducir lo que queda. Además asegúrate de que la cita sea
correcta, Jerry Banks aparece dos veces en tus referencias}



% \change[inline]{Ejemplo: Modelos de Montecarlo y modelos con ecuaciones
% diferenciales} \change[inline]{Un tipo de modelo son los modelos dinámicos
% discretos.}
Actualmente hay muchos tipos distintos de sistemas de simulación, como por
ejemplo los “Modelos de Montecarlo” y “Modelos en ecuaciones diferenciales”. No
obstante, el principal objeto de estudio de este proyecto será una categoría muy
importante a la que se conoce como “Modelos de simulación dinámicos discretos”,
especificamente aquellos que se pueden modelar a través de un diseño orientado a
eventos.

% \change[inline]{En realidad se han implementado ya algunos lenguajes
% específicos para simulación... (hablaremos de esto en los antecedentes)}
% \change[inline]{Aplicaciones de la simulación de sistemas (sistemas de salud,
% de transporte, control de inventarios, líneas de montaje o ensamblaje)} %
% Igual esto se va a contexto
Sin embargo, es necesario recalcar que existe ya una múltitud de aplicaciones
orientadas totalmente a la simulación de sistemas: Arena, AutoMod, Extend, entre
otros.\change{Tienes que hacer una cita aquí, esto no lo estás sacando de la nada.}
El hecho de que exista software específico para esto nos da un indicativo de lo
importante ue es esta área para el estudio de resultados y el modelado de
procesos. De hecho, el tipo específico de modelos del cual hablaremos es
muy utilizado en sistemas de salud, de transporte, control de inventarios,
líneas de montaje o ensamblaje, entre otros. \change{No olvides citar esta lista
de aplicaciones.}



% Bloque de compilación \change[inline]{Habla aquí de los procesadores del
% lenguaje (formal)} \change[inline]{¿Cuándo usar un procesador de lenguaje
% formal?}
Como mencionábamos antes, ya hay lenguajes de programación enfocados a la
simulación de sistemas. \change{Has dicho software, pero en realidad sí que hay
lenguajes dedicados a esto} Lo cual nos indica la unión de este campo con el de
compilación y diseño de procesadores de lenguajes formales. Ahora mismo parece
ser el momento más idóneo para usar un procesador de este tipo ya que debemos
seguir una serie de tokens y una gramática fija en estos lenguajes.

% \change[inline]{Herramientas de desarrollo de compiladores: Flex y Bison}



% Bloque de unión de tecnologías
% \change[inline]{Ambas tecnologías no parecen tener nada en común. Pero podemos
% dar un ejemplo de cuándo se pueden usar ambas a continuación}
% \change[inline]{Sí que se da la necesidad de una herramienta que trabaje con ambos}
Se puede ver, por tanto, que la unión entre los campos de compilación y
simulación de sistemas puede generar herramientas diseñadas específicamente para
agilizar el proceso de creación e implementación de modelos sin preocuparse
mucho por detalles irrelevantes para el programador de la simulación.

% \change[inline]{En este capítulo hablaremos de los conceptos básicos, el
% planteamiento y lo objetivos del proyecto.}
Sabiendo todo esto, procederemos hablar en este capítulo sobre la propuesta de
nuestro proyecto, sus objetivos y una serie de conceptos básicos necesarios para
la simulación de sistemas y la compilación en general. \improvement{Un poco seco,
cambia la redacción}



% Secciones aquí
\section{Motivación y planteamiento del proyecto}\label{sec:motivacion}

% Bloque simulación de sistemas
% \change[inline]{En simulación de sistemas, Un tipo de modelo son los modelos
% dinámicos discretos.}
% \change[inline]{Explicación de lo que son básicamente. (Igual una referencia a
% la definición de "dinámicos" y a la de "discretos")}
% \change[inline]{Estos tipos de modelos se suelen modelar usando diseños basados
% en eventos a través de grafos de sucesos.}
% \change[inline]{Estos modelos se suelen simular con un diseño basado en eventos
% o sucesos. Que vendría a ser...}
En el área de simulación de sistemas, una familia de modelos son los “modelos
dinámicos discretos”, siendo estos “dinámicos” porque su comportamiento cambia
en el tiempo y “discretos” porque estos cambios ocurren en instantes
determinados en vez de ocurrir continuamente.\improvement{Siento que estás
repitiéndote mucho, tienes que cambiar la calidad de este discurso}
Dichos sistemas se suelen modelar usando un diseño “basado en eventos”, el cual
nos indica que los cambios del modelo se deben considerar como “eventos” que
pueden ocurrir y dar lugar a otros eventos. Para ello, es común generar un
“grafo de sucesos” que represente todos los eventos posibles que pueden ocurrir
en el sistema y las relaciones que tienen entre estos.

% \change[inline]{A pesar de que cada modelo de este tipo tiene sus
% especificidades, todos estos comparten elementos en común independientes del
% sistema simulado (reloj, temporizador de eventos, lista de sucesos, etc.).}
% \change[inline]{Para poder implementar cada modelo, por tanto, es necesario
% añadir dichas implementaciones aparte de las propias del modelo. Sin embargo,
% sabiendo que todas éstas son compartidas, no se le debería dar al programador la
% responsabilidad de hacerlo si se pueden generar ya de antemano.}
Podríamos decir que cada modelo tendrá sus peculiaridades y diferencias
específicas a la hora de generar su implementación. Sin embargo, se da el hecho
de que todos los modelos de esta familia comparten elementos en común
independientemente del sistema a simular: el reloj de la simulación, el
temporizador de eventos, la lista de sucesos, entre otros. Por tanto, podemos
abstraer el desarrollo de estos programas de tal forma que el desarrollador sólo
deba encargarse de implementar todo aquello que sea único del modelo, quitándole
así la responsabilidad de generar los elementos en común y agilizando el
desarrollo en el proceso.




% Bloque de compilación
% \change[inline]{Los compiladores son usados...}
% \change[inline]{Los diseños basados en eventos en realidad cumplen con esta
% peculiaridad}
% \change[inline]{Podemos generar un analizador léxico, sintáctico y semántico
% que nos permita escribir de manera más eficaz nuestro modelo para luego
% convertirlo a algo ejecutable.}
% \change[inline]{Nosotros lo transformaremos a código Python, por lo tanto
% crearemos un transpilador.}
Tomando en cuenta que la principal herramienta de diseño de estos modelos serán
los grafos de sucesos, nos encontraremos con el hecho de que éstos siguen una
estructura representable a través de una gramática independiente de contexto.
Por tanto, es posible generar una serie de analizadores léxico, sintáctico y
semánticos que nos permitan procesar un lenguaje formal a otro código fuente.
Por esta razón hemos considerado una buena opción el uso de herramientas de
generación de compiladores como lo son Flex y Bison. Vemos que es posible crear
estos analizadores de forma que este hipotético nuevo lenguaje sea traducido a
código Python.

% \urgent{Generar una mejor explicación del problema encontrado. Hay que vender este TFG}

% El desarrollo de simuladores de sistemas dinámicos discretos, a pesar de que
% puede realizarse a mano, es capaz de llegar a resultar tedioso cuando se deben
% construir múltiples modelos distintos. Sin embargo, casi siempre se tendrá que
% estos sistemas harán uso de varias implementaciones que no dependen del
% simulador como tal.

% Como todas estos programas compartirán una serie de elementos en común (lista
% de eventos, reloj de la simulación, temporizador del simulador, entre otros)
% independientemente del modelo implementado, se plantea crear un nuevo lenguaje
% de programación orientado al desarrollo rápido de sistemas de simulación
% dinámicos discretos, creando para ello también un transpilador que traduzca
% estas implementaciones a código Python.
\section{Antecedentes}\label{sec:antecedentes}

\subsection{TFG del estudiante de la ETSIIT de la UGR (generador de sistemas de
simulación, nombre de sección provisional)}

\subsection{Lenguajes de simulación}

\change[inline]{Mencionar que hay dos categorías de lenguajes de programación
para desarrollar los simuladores: específicos y generales. El nuestro será
específico.}
\change[inline]{Aquí mencionaré algunos lenguajes de programación creados con el
fin de desarrollar simuladores}

\subsection{Software de simulación}

\change[inline]{Aquí mencionaré algunas aplicaciones creadas para desarrollar
simulaciones (como Arena)}

\cleardoublepage % End Chapter