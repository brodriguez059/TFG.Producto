\section{Alcance}

% \urgent[inline]{El alcance debe explicar aquellas cosas que planteas hacer
% para el proyecto. Lo que podrías llamar como “tareas”. Procura explicar de
% manera detallada todo esto}

% \change[inline]{Vamos a crear un nuevo lenguaje enfocado a la generación de
% modelos de sistemas dinámicos discretos.}

El alcance de este proyecto incluye el trabajo necesario para diseñar,
implementar y documentar los analizadores léxico, sintáctico y semántico de un
nuevo lenguaje de programación enfocado en la agilización del desarrollo de
modelos de simulación dinámicos discretos basados en eventos.

% \change[inline]{Dicho lenguaje lo traduciremos a Python, por lo que debemos
% crear un transpilador.} \change[inline]{Usaremos las herramientas de
% desarrollo de compiladores Flex y Bison}

Dicho lenguaje será convertido a código Python, por lo que se incluirá una serie
de funciones y módulos adicionales que funcionarán a modo de librería básica del
programa una vez traducido éste. Por tanto, para poder generar este
transpilador, haremos uso de las herramientas de desarrollo de compiladores Flex
y Bison.

% \change[inline]{Usaremos Kanban para planificar y gestionar el proyecto}

Por último, la gestión del proyecto se realizará siguiendo la metodología de
desarrollo ágil Kanban para permitir construir de manera iterativa las distintas
funcionalidades del proyecto. \improvement{Un poco crudo, ¿quizás puedes decir
un poco más al respecto?}

\section{Objetivos}

Conocido ya el alcance, podemos proceder a dar la lista de objetivos principales
del proyecto: \improvement{Así como lista no me convence mucho, revisa si puedes
convertirlo a subsecciones con nombres relevantes (gestión, diseño,
desarrollo...)}

\begin{itemize}
    \item Hacer uso del método Kanban para el desarrollo y gestión del proyecto.

    \item Diseñar un nuevo lenguaje de programación inspirado en el léxico y
    sintaxis de Flex, Bison y Python. Dicho lenguaje nos permitirá implementar
    simuladores de sistemas dinámicos discretos cuyos diseños estarán basados en
    eventos, pero se podrán implementar otros tipos de diseños a través de
    estos. Se debe permitir la rápida implementación de este tipo de modelos a
    través de grafos de sucesos.

    \item Hacer uso de las herramientas de generación de compiladores Flex y
    Bison con el fin de construir los analizadores léxico, sintáctico y
    semántico del nuevo lenguaje.

    \item Permitir que la traducción del lenguaje incluya dentro del fichero
    generado las estructuras de datos, funciones y procedimientos que tienen en
    común todos los sistemas dinámicos discretos:
    \begin{itemize}
        \item Generadores de datos aleatorios para distintos tipos de
        distribuciones.
        \item Reloj y temporizador de simulación para ejecutar los eventos.
        \item Estructura de datos para almacenar los sucesos según deben ocurrir
        en el tiempo.
        \item Las respectivas implementaciones mínimas de los dos eventos que
        siempre formarán parte de todos los modelos: “Inicio” y “Fin”.
        \item Generador de informes final que se ejecutará al finalizar la
        simulación y mostrará los resultados que se deseaban estudiar con ésta.
    \end{itemize}

    \item Permitir que el programador del lenguaje se encarge sólo de realizar
    las implementaciones pertinentes al sistema de simulación que desee
    desarrollar:
    \begin{itemize}
        \item Especificación de las variables globales, variables de entrada,
        contadores estadísticos y medidas de rendimiento propias del modelo.
        \item Inclusión de eventos adicionales y sus acciones correspondientes.
        \item Creación y eliminación de eventos en función de tiempo y
        condicionales lógicas.
        \item Inclusión de código adicional escrito directamente en Python en
        caso de ser necesario.
    \end{itemize}

    \item Permitir la configuración del comportamiento del programa traducido a
    través de parámetros de entrada.
\end{itemize}